%!TEX root = ..\..\main.tex
\chapter{Conclusion to Part \ref{part:optimization for stats}}
\label{Ch:StatisticalOptimizationConclusion}

\lhead{Chapter \ref{Ch:StatisticalOptimizationConclusion}. \emph{Conclusion}}

\graphicspath{{Figures/Mixtures/}}

%-------------------------------------------------------------------------------
%	Chapter Text
%-------------------------------------------------------------------------------

The second part of this thesis has focused on two optimization problems that arise in statistical inference. A property shared by these two problems is that in both of them we search for a discrete probability distribution and find that our solution contains surprisingly few points of support. While we were able to find and prove some new results concerning this phenomenon for maximum likelihood location mixtures, empirical results suggested that these could be either improved or generalised. Concerning our deconvolution problem, we were unable to prove any theoretical results, but again empirical exploration indicated that this phenomenon was typical and that there could be underlying reasons behind why it occurs. We therefore think that there is potential for future work to significantly add to the content of the second part of this thesis. 

We would also like to note that these are not the only scenarios in which a phenomenon like this occurs. For a third example, we may consider extremal Markov moment problems (see \cite{Krein1977-ak} for more information) in which we find that the distribution which minimizes a certain objective, subject to certain constraints, has a known number of point masses which depends on the constraints. This is a well understood problem, and gives us hope that a deeper understanding of both our deconvolution problem, and the maximum likelihood mixture probem, may be acheivable. In addition to this, the existence of this phenomenon in a variety of different settings hints at the possibility of a broad general theory connecting them.

Finally, there are also some other more specific areas in which future research would be of interest. Currently, characteristic functions and phase functions are poorly understood, particularly concerning their decomposability. 
% In Chapter \ref{Ch:Deconvolution}, we required that $U$ had a real-valued and non-negative characteristic function. While a real-valued characteristic function corresponds to a symmetric distribution, we do not know what properties distributions with non-negative characteristic functions possess. In our experience, we were unable to find any real-valued and non-negative characteristic functions that did not correspond to unimodal distributions, but we were unable to determine if this was true of all real non-negative characteristic functions. 
We do not know when and if there is a unique indecomposable characteristic function with a given phase function. Answers to this question would help us understand better when our deconvolution method works best, and potentially provide insight into how we think about convolved distributions.


[ANYTHING ELSE?]