%!TEX root = ..\main.tex
\chapter{Introduction to this Thesis}
\label{Ch:ThesisIntro}

\lhead{Chapter \ref{Ch:ThesisIntro}. \emph{Introduction to this Thesis}} % This is for the header on each page

Initially, this thesis was intended to be made up entirely of the contents of Part \ref{part:optimization for stats}, along with what we hoped would be several significant further contributions to the study. However, the practicalities of a deadline, along with the challenging nature of the research, meant that the decision was made to augment this thesis with an essentially separate section of study. This is what makes up Part \ref{part:coupling time}.

% This thesis is made up of two parts. Part \ref{part:coupling time} titled, \nameref{part:coupling time} and part \ref{part:optimization for stats} titled, \nameref{part:optimization for stats}. 
The reader should view these two parts as standalone topics, to be read independently. However, they are not without any commonality. Both are within the realm of stochastic mathematics, Part \ref{part:coupling time} being a study of a random variable constructed from a stochastic process, and Part \ref{part:optimization for stats} being a study of probability distributions that maximize certain statistical objective functions. 

We note that while a typical thesis is a cohesive whole, this is certainly not a requirement to obtaining a doctorate. Indeed, to obtain the degree of Doctor of Philosphy, the candidate is required to produce a substantial piece of original research. We believe that the sum of the research contained in both parts of this thesis is sufficiently substantial, and thus adequate for submission.

