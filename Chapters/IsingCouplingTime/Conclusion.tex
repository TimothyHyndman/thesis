%!TEX root = ..\..\main.tex
\chapter{Conclusion to Part \ref{part:coupling time}}
\label{Ch:CouplingConclusion}

\lhead{Chapter \ref{Ch:CouplingConclusion}. \emph{Conclusion}} % This is for the header on each page

In the first part of this thesis, we have examined the dynamics of the continuous-time Ising heat-bath Glauber process. We started by describing the discrete time dynamics, as well as the two coupled chains from which we define the coupling time. We then described how to `continuize' the discrete time dynamics. It is the distribution of the coupling time of these continuous time dynamics that was the central object of our analysis. We proved that the asymptotic distributions of the discrete coupling time and the continuous coupling time are equivalent via Proposition \ref{prop:discrete vs continuous distributions} which is applicable in much more generality. This was important to show since our analysis focuses on the continuous-time dynamics, whereas our motivation for the study of the coupling time was that the discrete coupling time has the same distribution as the `Coupling From the Past' algorithm. We ended the introduction by giving a description of information percolation, as used by Lubetzky and Sly in \cite{Lubetzky2016-wd}, and of compound Poisson approximation, as described in \cite{Barbour2001-nh}.

In Chapter \ref{Ch:1D}, we looked at the coupling-time on the cycle. Our main result was that at any inverse-temperature $\beta$, the coupling time converges to a Gumbel distribution. This confirmed, for $d = 1$, a conjecture in \cite[Conjecture 7.1]{Collevecchio2018-nq} that the coupling time of the Ising heat-bath process on the lattice, $G_L = (\mathbb{Z} / L \mathbb{Z})^d$, converges to a Gumbel distribution for all $\beta < \beta_c$. We used the two techniques of information percolation, and compound Poisson approximation to create the proof.

In Chapter \ref{Ch:GeneralResults}, we extended the class of graphs we considered from the cycle, to a family of vertex-transitive graphs with fixed degree. We showed that as well as the $d$-dimensional lattices, this family also included other classes of graphs which are of practical interest with respect to the Ising model. Our main result was that for sufficiently small $\beta$, the coupling time converges to a Gumbel distribution. This partially confirms, for $d > 1$, the afore mentioned conjecture in \cite{Collevecchio2018-nq}. While our result does not hold up to the critical temperature as conjectured, it holds for a larger family of graphs and we commented that not holding up to the critical temperature on lattices is an expected consequence of the increased generality of graphs to which our result is applicable.

Of particular note is that the main part of the proof in which we require $\beta$ to be small concerns an object which is almost identical to a quantity used by Lubetzky and Sly in \cite{Lubetzky2015-po}. In this paper, Lubetzky and Sly proved the existence of cutoff on a wide class of transitive graphs for $\beta$ small enough. They later refined this proof in \cite{Lubetzky2016-wd} to apply up to the critical temperature on the $d$-dimensional lattices. In \cite{Lubetzky2017-nc} they were able to specify how small $\beta$ must be for cutoff to occur on any sequence of graphs with maximum degree $\Delta$.

This encourages future efforts to both refine our proof on the $d$-dimensional lattices so that it holds up all the way up to the critical $\beta$, and to specify a sufficiently small value of $\beta$ for our result to hold in general.
We would also like to investigate other problems for which information percolation can provide solutions. The success of information percolation in establishing cutoff (a previously notoriously difficult problem), and our success in applying it to our problem, suggests that it is a powerful and flexible tool for analysing spin dynamics. Given that it is a relatively recent technique, we suspect that there exists a variety of new results which can be obtained using information percolation.

[ANYTHING MORE SPECIFIC?]