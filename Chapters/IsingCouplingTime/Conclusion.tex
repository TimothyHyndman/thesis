%!TEX root = ..\..\main.tex
\chapter{Conclusion to Part \ref{part:coupling time}}
\label{Ch:CouplingConclusion}

\lhead{Chapter \ref{Ch:CouplingConclusion}. \emph{Conclusion}} % This is for the header on each page

In the first part of this thesis, we have examined the dynamics of the continuous-time Ising heat-bath Glauber process. We started by describing the discrete time dynamics, as well as the two coupled chains from which we define the coupling time. We then described how to `continuize' the discrete time dynamics. It is the distribution of the coupling time of these continuous time dynamics that was the central object of our analysis. We proved that the asymptotic distributions of the discrete coupling time and the continuous coupling time are equivalent, using Proposition \ref{prop:discrete vs continuous distributions} which is applicable in much more generality. This was important to show, since our analysis focuses on the continuous-time dynamics, whereas our motivation for the study of the coupling time was that the discrete coupling time has the same distribution as the random running time of the `Coupling From the Past' algorithm. We ended the introduction by giving a description of information percolation, as used by Lubetzky and Sly in \cite{Lubetzky2016-wd}, and of compound Poisson approximation, as described in \cite{Barbour2001-nh}.

In Chapter \ref{Ch:1D}, we looked at the coupling-time on the cycle. Our main result was that at any inverse-temperature $\beta$, the coupling time converges to a Gumbel distribution. This confirmed, for $d = 1$, a conjecture in \cite[Conjecture 7.1]{Collevecchio2018-nq} that the coupling time of the Ising heat-bath process on the torus, $G_L = (\mathbb{Z} / L \mathbb{Z})^d$, converges to a Gumbel distribution for all $\beta < \beta_c$. We used the two techniques of information percolation, and compound Poisson approximation to construct the proof.

In Chapter \ref{Ch:GeneralResults}, we extended the class of graphs considered from the cycle to a family of vertex-transitive graphs with fixed degree. We showed that as well as the $d$-dimensional tori, this family also included other classes of graphs which are of practical interest with respect to the Ising model. Our main result was that for sufficiently small $\beta$, the coupling time converges to a Gumbel distribution. This partially confirms, for $d > 1$, the afore mentioned conjecture in \cite{Collevecchio2018-nq}. While our result does not hold up to the critical temperature as conjectured, it holds for a larger family of graphs.
% and we commented that not holding up to the critical temperature on lattices is an expected consequence of the increased generality of graphs to which our result is applicable.

Of particular note is that the main part of the proof in which we require $\beta$ to be small concerns an object which is almost identical to a quantity used by Lubetzky and Sly in \cite{Lubetzky2015-po}. In \cite{Lubetzky2015-po}, Lubetzky and Sly proved the existence of cutoff on a wide class of transitive graphs for $\beta$ small enough. They later refined this proof in \cite{Lubetzky2016-wd} to apply up to the critical temperature on $d$-dimensional lattices. In \cite{Lubetzky2017-nc} they were able to specify how small $\beta$ must be for cutoff to occur on any sequence of graphs with maximum degree $\Delta$.

This encourages future efforts to both refine our proof on the $d$-dimensional lattices so that it holds up all the way up to the critical $\beta$, and to specify a sufficiently small value of $\beta$ for our result to hold in general.
We would also like to investigate other problems for which information percolation can provide solutions. Information percolation has not only been successful in establishing cutoff for the Glauber dynamics for the Ising model, but has also recently been applied to cutoff for the Swendsen-Wang dynamics on the lattice \cite{Nam2018-io} and cutoff for the random-cluster model \cite{Ganguly2018-lv}. We have shown that information percolation can also establish results concerning the coupling time. 

Given that it is a relatively recent technique, we suspect that there exists a variety of new results which can be obtained using information percolation. In particular, in \cite[Conjecture 2.7]{Collevecchio2018-nq}, it is conjectured that, for generic choices of parameters, the coupling time for the FK heat-bath coupling for the random-cluster model on $d$-dimensional lattices converges to a Gumbel distribution. Just as the coupling time for the Ising heat-bath coupling corresponds precisely to the coupon collector's problem when $\beta = 0$, there is also a special case in which the coupling time for the FK heat-bath coupling corresponds precisely to the coupon collector's problem (which has a known Gumbel limit). We expect that this Gumbel limit holds for a range of parameter values in addition to this special case, just as we observed for the Ising heat-bath dynamics.

% [ANYTHING MORE SPECIFIC?]

% [NOT JUST SPIN DYNAMICS. SLY HAD AN ARXIV PAPER MID 2018 DOING SWENDSEN WANG AND THERE WAS AN ARXIV PAPER DEC 2018 DOING RANDOM-CLUSTER MODEL. WE HAVE GUMBEL CONJECTURES FOR RANDOM-CLUSTER - CAN INFO PERCOLATION PROVE THEM??]

% [I ALSO TALK HOW GENERAL YOU THINK THE GUMBEL RESULT IS -> EXPECT IT TO HOLD ALSO FOR RANDOM-CLUSTER MODEL]