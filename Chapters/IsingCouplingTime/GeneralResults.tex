%!TEX root = ..\..\main.tex
\chapter{The Coupling Time on Vertex Transitive Graphs}
\label{Ch:GeneralResults}

\lhead{Chapter \ref{Ch:GeneralResults}. \emph{The Coupling Time on Vertex Transitive Graphs}}

For the most part, this chapter will be similar in structure and content to Chapter \ref{Ch:1D}. The main difference is that we extend the family of graphs on which we consider the Ising heat-bath Glauber dynamics from the cycle to a certain family of vertex transitive graphs. Again, the main result, Theorem \ref{thm:Coupling Distribution on transitive graph}, concerns the coupling time, $T_n$, as defined in Section \ref{sec:the coupling time}, and establishes that at sufficiently high temperature (that is, for $\beta$ small enough), the coupling time converges in distribution to a Gumbel distribution.

Restricting $\beta$ to be sufficiently small is a consequence of the increased generality of this chapter. As mentioned in Chapter \ref{Ch:1D}, in the high-temperature regime we expect the dynamics to be similar to those when $\beta = 0$. When $\beta = 0$ the problem simplifies to the coupon collector's problem, which is known to have a Gumbel limit. However, at the critical temperature, and below in the low-temperature regime, there is no reason to suspect that the dynamics will behave similarly to when $\beta = 0$. So our restriction of $\beta$ to be small enough for the result to hold is, on at least a descriptive level, somewhat expected. 

Our result partially confirms the conjecture by \citeauthor{Collevecchio2018-nq} in \cite{Collevecchio2018-nq} that the coupling time for the Ising heat-bath process on the $d$-dimensional lattice, $G_L = (\mathbb{Z}/L\mathbb{Z})^d$, converges to a Gumbel distribution as $L \rightarrow\infty$ for all $\beta < \beta_C$. Our result does not hold all the way up to the critical temperature. This is due to the fact that we are considering a larger class of graphs than just the square lattice, and so it is unreasonable to expect such a result to provide sharp bounds for the lattice. A separate treatment of the square lattice in particular may be needed for a result holding all the way up to the critical temperature. Note that this is what \citeauthor{Lubetzky2013-yv} did in \cite{Lubetzky2013-yv} to prove the existence of cutoff for the full high-temperature regime. Since our proof is also based on information percolation there is good reason to think that a similar approach could also work to extend our result all the way to the critical temperature.

The main part of our proof which requires $\beta$ to be sufficiently small is in Lemma \ref{lem: prob intersect given X_i = 1}. In particular, this Lemma concerns a quantity which is closely related to a quantity used in \cite{Lubetzky2015-po} to prove the existence of cutoff (see the comments immediately preceding Lemma \ref{lem: prob intersect given X_i = 1}). This similarity further encourages future efforts to sharpen our result.

To state the main result we must first define the graphs on which it is valid. Let $(G_{n})$ be a sequence of vertex-transitive graphs with fixed degree $\Delta$ and where $n$ counts the number of vertices in $G_n$. Let $P_n(k)$ denote the number of vertices at distance $k$ from a vertex $i$ in $G_n$ and let $Q_n(k)$ denote the number of vertices at distance $k$ or less from a vertex $i$ in $G_n$. Define
\begin{equation}
	\mathscr{G} = \left\{(G_n) : \exists C_2 > 0 \text{ such that} \limsup_{n \rightarrow \infty}\sum_{k = 1}^\infty P_n(k)\euler^{-k} \leq C_2, Q_n(\ln^2(n)) = o(n) \right\}.
	\label{eq:set valid graphs}
\end{equation}
In this chapter we consider sequences of vertex-transitive graphs $(G_n) \in \mathscr{G}$. 

It is worth verifying that the set $\mathscr{G}$ contains some graph sequences that are of interest. We start by showing that it contains the $d$-dimensional discrete tori.

\begin{lemma}
	Define the sequence of length $L$ $d$-dimensional square lattices on a torus, $(G_L)_{L \geq 1}$, via $G_L = (\mathbb{Z}/L \mathbb{Z})^d$. Then $(G_L)_{L \geq 1} \in \mathscr{G}$.
\end{lemma}
\begin{proof}
	The torus $G_L = (\mathbb{Z}/L \mathbb{Z})^d$ is obviously vertex transitive and the sequence $(G_L)_{L \geq 1}$ has fixed degree $\Delta = 2d$. By definition $P_n(k) \leq Q_n(k)$ and we can upper bound $Q_n(k)$ by the number of vertices contained within distance $k$ of the origin on the infinite $d$-dimensional integer lattice, $Q_\infty(k)$.

	To bound $Q_\infty(k)$, consider first the number of vertices contained within distance $k$ of the origin in the closed positive orthant, $Q_\infty(k)^+$. That is, the number of vertices $\vect{x} = (x_1, \dots, x_d)$ with $x_i \geq 0$ and such that
	\begin{align}
		\sum_{i = 1}^d x_i \leq k
	\end{align}
	We note that we can represent each vertex in the positive orthant within distance $k$ of the origin by $k$ stars separated by $d$ bars, and taking $x_i$ to be the number of stars between bars $i-1$ and $i$. So for example, if $k = 5$ and $d = 3$, the arrangement
	\begin{equation}
		\star | \star \star | | \star \star
	\end{equation}
	represents the vertex $\vect{x} = (1, 2, 0)$. Note that any stars after the last bar are not included (this accounts for $\vect{x}$ being closer than distance $k$ to the origin). There are $k + d$ choose $d$ ways to arrange the stars and bars and so
	\begin{equation}
		Q_\infty(k)^+ \leq \binom{k+d}{d}
	\end{equation}
	and since in $d$ dimensions there are $2^d$ orthants,
	\begin{equation}
		P_n(k) \leq Q_n(k) \leq Q_\infty(k) \leq \frac{2^d}{d!} (k+1)(k+2)\dots(k+d).
	\end{equation}
	This is a degree $d$ polynomial in $k$ which satisfies the constraints in \eqref{eq:set valid graphs}.
\end{proof}

Another class of graphs contained in $\mathscr{G}$ are the $m$th powers of $G_L = (\mathbb{Z}/L \mathbb{Z})^d$.	The $m$th power of a graph $G$ is a graph, $G^m$, with the same vertices as $G$, but in which we make adjacent all vertices whose distance in $G$ is no more than $m$. Clearly, $Q_n^{G^m}(k) \leq Q_n^G(mk)$, and $G^m$ inherits the transitivity of $G$. Hence sequences of powers of $G_L = (\mathbb{Z}/L \mathbb{Z})^d$ are also in $\mathscr{G}$. This is of practical interest since the Ising model on $G^m$ corresponds to the Ising model with $m$ nearest neighbours on $G$ (see for example \cite{Domb1951-nb} on the Ising model with next nearest neighbours).

We now define a couple of quantities. 
% Recall from \eqref{eq:magnetization on volume definition} that given a fixed configuration, the magnetization on a volume $\Lambda$ is the sum of the spins on $\Lambda$. We can consider an analogous random variable, $M_t$, that is the sum of the spins of the top process, $\mathcal{T}_t$, at any fixed time $t$. 
%
% analogous random variable $M_t$ which measure
%
Firstly, the \emph{expected magnetization at vertex $i$ at time $t$} is
\begin{equation}
	 	m_t(i) = \expect[\mathcal{T}_t[i]]
\end{equation} 
where $(\mathcal{T}_t)_{t \geq 0}$ is the dynamics starting from the all-plus configuration. Note that on transitive graphs, with which this chapter is concerned, we can drop the dependence on $i$ and simply write $m_t$ for the expected magnetization at any vertex at time $t$. 

This quantity is not to be confused with the total magnetization of a stationary Ising configuration, as defined in Section \ref{sec:the phase transition}. Recall from \eqref{eq:magnetization on volume definition} that given a fixed configuration, the total magnetization on a volume $\Lambda$ is the sum of the spins on $\Lambda$. We could consider an analogous random variable, $M_t$, that is the sum of the spins of the top process, $\mathcal{T}_t$, at any fixed time $t$. Then we have
\begin{equation}
	\frac{\expect[M_t]}{n} = m_t.
\end{equation}
As we do not need to refer to the total magnetization of a stationary Ising configuration in this chapter, we will refer to the expected magnetization as simply the magnetization.

% This quantity is not to be confused with the magnetization on a volume (as mentioned in Section \ref{sec:the phase transition}) which has nothing to do with dynamics. The magnetization on a volume is a random variable given by the sum of all the spins on a given volume where the spins are distributed according to the Gibbs measure. In contrast, the magnetization at a vertex, as defined here, is a deterministic function of time that is a property of the Glauber dynamics.

We can now define the time
\begin{equation}
	\label{eq:definition t_c(n)}
	t_c(n) = \inf\left\{ t > 0 : m_t = \frac{1}{n}\right\}.
\end{equation}
which is around the time it takes for the top and bottom chains to couple. Note that this is well defined since by definition $m_0 = 1$ and by Lemma \ref{lemma: exponential decay magnetization}, $m_t$ is continuous and asymptotically decays to zero, and hence must take value $1/n$ at some time.

One way of interpreting this time is that at time $t_c(n)$, $\expect[M_{t_c(n)}] = 1$. By way of comparison, at the initial state, $\expect[M_0] = n$, and at stationarity, $\expect[M] = 0$.

% [SHOW EXISTS t S.T. $m_t = 1/n$, SEE TIM'S COMMENT]

\begin{theorem}
\label{thm:Coupling Distribution on transitive graph}
	Let $(G_n) \in \mathscr{G}$ be a sequence of vertex-transitive graphs, $G_n = (V, E)$ with $n = |V|$ vertices. Let $T_n$ be the coupling time for the continuous-time Ising heat-bath dynamics for the zero-field ferromagnetic Ising model on $G_n$. Then for any small enough inverse-temperature $\beta$, there exists a subsequence $(T_m)$ of $(T_n)$ such that
	\begin{equation}
		\lim_{m \rightarrow \infty} \prob[T_m < z + t_c(m)] = \euler^{-C_1\euler^{-C_2 z}}
	\end{equation}
	for some
	\begin{equation}
		C_1 = C_1(\beta, \mathscr{G}) \in \left(0,1\right]
	\end{equation}
	and
	\begin{equation}
		C_2 = C_2(\beta, \mathscr{G}) \in \left[1 - \beta \Delta, 1\right].
	\end{equation}

\end{theorem}

The proof of Theorem \ref{thm:Coupling Distribution on transitive graph} will be given in Section \ref{sec:proof thm coupling nd} after the essential preliminaries are presented. In Section \ref{sec:information percolation in higher dimensions} we describe an alternative construction of the histories that can sometimes be easier to work with, and state some results concerning the magnetization. Then in Section \ref{sec:nd problem setup}, we outline the overall approach to the proof. The method is similar to the method used in Chapter \ref{Ch:1D}. Finally, we defer results directly concerning the update histories to Section \ref{sec:additional lemmas nd}.

\section{Information percolation in higher dimensions}
\label{sec:information percolation in higher dimensions}
In the previous chapter, we showed that on the cycle, there was a coupling that made the update history of a single vertex to be a continuous-time random walk that died at rate $\theta$. On lattices of dimension $d > 2$, we can no longer use this coupling and so the updates histories are significantly more complex. 

Recall from Section \ref{sec: definition update support function} that given a target time $t_*$, the update support of a vertex set $A$ at time $t$, $\mathcal{H}_A(t)$, is the set of vertices whose spins at time $t$ determine the spins of $A$ at time $t_*$. This support can be constructed using the update sequence along $(t, t_*]$. Developing the update support backwards in time from $t = t_*$ produces a subgraph of $V \times [0, t_*]$ which we write as $\mathcal{H}_A$ and call the update history of $A$.

It is not immediately obvious what these histories look like. To gain some intuition, and to simplify some later proofs, we will construct a different subgraph, $\hat{\hist}_A$ which contains $\hist_A$. In doing so we also define an analogous update support $\hat{\hist}_A(t)$ for every $t \in [0, t_*]$. 

Construct $\hat{\hist}_A$ as follows: For each $i \in A$, create a temporal edge between $(i, t_*)$ and $(i, t_i)$ where $t_i$ is the time of the latest update to $i$ (or $0$ if $i$ is never updated). Then for each update $(i, U, t_i)$, we either terminate the edge if $U$ is such that the update is oblivious, or we add spatial branches to each of the neighbours of $i$. We repeat this process recursively for the neighbours of $i$ until every branch has been terminated due to an oblivious update or has reached time $0$. Since oblivious updates do not depend on any other vertices, and since a non-oblivious update to $i$ depends on at most the neighbours of $i$, we have that $\hist_A(t) \subseteq \hat{\hist}_A(t)$, and that $\hist_A$ is a subgraph of $\hat{\hist}_A$.

Note that it is possible for vertices to be removed from $\mathcal{H}_A(t)$ by updates that are not oblivious (see Figure \ref{fig:nonoblivious shrink} for an example on the cycle). Since our method for constructing $\hat{\hist}_A$ does not take this into account, in general we expect that $\hist_A$ and $\hat{\hist}_A$ are not equal.

% To ensure a distinction between the two, the history that results from the above construction we will denote $\hat{\mathcal{H}}_A$, and likewise $\hat{\mathcal{H}}_A(t)$ for the history at time $t$ that results from the above construction. We have that
% \begin{equation}
% 	\mathcal{H}_A(t) \subseteq \hat{\mathcal{H}}_A(t)
% \end{equation}
% and also that $\mathcal{H}_A$ is a subgraph of $\hat{\mathcal{H}}_A$.

\subsection{The magnetization}
One quantity which we used multiple times in Chapter \ref{Ch:1D} was $\prob[X_i = 1] = \prob[\hist_i(0) \neq \emptyset]$. Although it was not required earlier, we would now like to make clear that this is in fact the magnetization at time $t_*$. 

Recall that the magnetization at vertex $i \in V$ at time $t \geq 0$ is defined to be
\begin{equation}
	m_t(i) = \expect[\mathcal{T}_t[i]],
\end{equation}
where $(\mathcal{T}_t)_{t \geq 0}$ is the dynamics starting from the all-plus configuration. %On vertex-transitive graphs, we can drop the vertex specific notation and simply write $m_t$. 
Given a monotonically coupled chain $(\mathcal{B}_t)_{t\geq0}$, starting in the all minus configuration and such that $\mathcal{T}_t[i] \geq \mathcal{B}_t[i]$ for all $t\geq 0$ and $i \in V$, we can split up this expectation by conditioning on the event $A_t = \{\mathcal{T}_t[i] \neq \mathcal{B}_t[i]\}$. We obtain that
\begin{align}
	m_t(i) &= \expect[\mathcal{T}_t[i]]\\
	&= \prob\left[A_t\right] \left(\prob\left[\mathcal{T}_t[i] = 1 | A_t\right] - \prob\left[\mathcal{T}_t[i] = -1|A_t\right]\right)  \\
	&\phantom{= } + \prob\left[A_t^\complement\right] \left( \prob\left[\mathcal{T}_t[i] = 1 |A_t^\complement\right] - \prob\left[\mathcal{T}_t[i] = -1|A_t^\complement\right] \right). \notag
\end{align}

Now if event $A_t^\complement$ holds, $\mathcal{T}_t[i] = \mathcal{B}_t[i]$, and so by symmetry vertex $i$ must take values $-1$ and $+1$ uniformly. Furthermore, by the monotonicity of our coupling, if $A_t$ holds, we must have that $\mathcal{T}_t[i] = +1$ and $\mathcal{B}_t[i] = -1$.
So
\begin{align}
	m_t(i) = \prob[A_t].
\end{align}
Finally, given a target time $t_*$, $X_i$ is defined such that $\{X_i = 1\} = A_{t_*}$. So 
\begin{equation}
	\label{eq:prob X_i = 1 is m_t*}
	\prob[X_i = 1] = m_{t_*}(i).	
\end{equation}
% This motivates the following restatement of part of Lemma 2.1 from \cite{Lubetzky2016-wd}.

% \begin{lemma}[\cite{Lubetzky2016-wd}, Lemma 2.1]
% 	There exist some constant $c_{\beta, d} > 0$ such that for any $t > 0$,
% 	\begin{equation}
% 		m_t \leq 2 \euler^{-c_{\beta, d} t}
% 	\end{equation}
% \end{lemma}
% \begin{corollary}
% 	\begin{equation}
% 		\prob[X_i = 1] \leq 2 \euler^{-c_{\beta, d} t_*}
% 	\end{equation}
% \end{corollary}

We end this section with some results concerning the magnetization, and in particular, the magnetization at time
\begin{equation}
\label{eq: define t star general graphs}
	t_* = t_c(n) + z.
\end{equation}
The following comes from \cite{Lubetzky2017-nc} and is valid on any graph, not just transitive ones.
\begin{lemma}[\cite{Lubetzky2017-nc}, Claim 3.3]
\label{lemma: exponential decay magnetization}
	On any graph with maximum degree $\Delta$, for any $t, s > 0$ we have
	\begin{equation}
		\euler^{-2s} \leq \frac{\sum_i m_{t+s}[i]^2}{\sum_i m_t[i]^2} \leq \euler^{-2(1 - \beta \Delta)s}.
	\end{equation}
\end{lemma}
The following corollaries are then straightforward.
\begin{corollary}
	\label{cor:exponential decay magnetization}
	On any vertex transitive graph with degree $\Delta$, for any $t, s > 0$ we have
	\begin{equation}
		\euler^{-s} m_t \leq m_{t+s} \leq m_t \euler^{-(1 - \beta \Delta)s}.
	\end{equation}
\end{corollary}
% \begin{corollary}
% \label{cor:magnetization of t star}
% 	On any vertex transitive graph with degree $\Delta$, $m_{t_*}$ can be bounded as follows:

% 	For $z \geq 0$,
% 	\begin{equation}
% 		\frac{\euler^{-z}}{n} \leq m_{t_*} \leq \frac{\euler^{-(1 - \beta \Delta)z}}{n}.
% 	\end{equation}

% 	For $z \leq 0$,
% 	\begin{equation}
% 		\frac{\euler^{-(1 - \beta \Delta)z}}{n} \leq m_{t_*} \leq \frac{\euler^{-z}}{n}.
% 	\end{equation}
% \end{corollary}
%NEW VERSION THAT TAKES INTO ACCOUNT <= instead of = in definition of t_c

% [FOLLOWING NEEDS LOWER BOUNDS FOR LAMBDA!!!!!]
\begin{corollary}
\label{cor:magnetization of t star}
	On any vertex transitive graph with degree $\Delta$, $m_{t_*}$ can be bounded as follows:

	For $z \geq 0$,
	\begin{equation}
		\frac{\euler^{-z}}{n} \leq m_{t_*} \leq \frac{\euler^{-(1 - \beta \Delta)z}}{n}.
	\end{equation}

	For $z \leq 0$,
	\begin{equation}
		 \frac{\euler^{-(1 - \beta \Delta)z}}{n} \leq m_{t_*} \leq \frac{\euler^{-z}}{n}.
	\end{equation}
\end{corollary}
Bearing in mind that $m_0 = 1$, we also obtain a bound on $t_c(n)$.
\begin{corollary}
\label{cor:bound t_c(n)}
	On any vertex transitive graph with degree $\Delta$, for $\beta < 1/\Delta$
	\begin{equation}
		\ln(n) \leq t_c(n) \leq \frac{\ln(n)}{1 - \beta \Delta}.
	\end{equation}
\end{corollary}
% \begin{corollary}
% 	On any vertex transitive graph with degree $\Delta$, for $\beta < 1/\Delta$
% 	\begin{equation}
% 		\ln(n) \leq t_c(n) \leq 
% 	\end{equation}
% \end{corollary}

\section{Problem set-up}
\label{sec:nd problem setup}
As in Chapter \ref{Ch:1D}, to prove Theorem \ref{thm:Coupling Distribution on transitive graph} we will use the method sketched out in Section \ref{sec:compound poisson overview}. The various quantities left unspecified there are defined almost identically to Chapter \ref{Ch:1D}; the main difference in construction being that here we are interested in graph sequences $(G_n) \in \mathscr{G}$ and we define $t_*$ as in \eqref{eq: define t star general graphs}.

% The overall approach is almost exactly as described in Section \ref{sec:1D problem setup}; at a time $t_*$ we count the number of vertices at which the bottom and top chains differ and show that the distribution of this random variable, which we call $W$, is close to an appropriately chosen compound Poisson distribution. The main difference in the construction is that we use a different $t_*$ here.

Recalling the definition of $t_c(n)$ in \eqref{eq:definition t_c(n)}, fix $z \in \mathbb{R}$ and set
\begin{equation}
	t_* = t_c(n) + z.
\end{equation}
By Corollary \ref{cor:bound t_c(n)}, for any fixed $z \in \mathbb{R}$, $t_* > 0$ for all sufficiently large $n$ and we only consider such $n$ in what follows. As in Chapter \ref{Ch:1D}, we define the vertex sets
\begin{align}
	B_i &= \{j\neq i : d(i,j) \leq b_n \},\\
	C_i &= \{j\notin B_i\cup \{i\}: d(i,j) \leq c_n \},\\
	D_i &= V \setminus (B_i \cup C_i \cup \{i\}),
\end{align}
where $b_n = \ln(n)$ and $c_n = \ln^2(n)$.
From here we define $X_i$, $U_i$, $W_i$, $\delta_1$, $\delta_4$, $\lambda$, and $\mu$ exactly as in Section \ref{sec:application of compound poisson} using our new definition for $t_*$. From \eqref{eq:bound distribution T}, we get the following Corollary of Theorem \ref{thm: compound poisson approximation}.

% For each vertex $i \in V$, define indicators
% 	\begin{equation}
% 		X_i = 
% 		\begin{cases}
% 			1 & \mathscr{B}_{t_*}[i] \neq \mathscr{T}_{t_*}[i],\\
% 			0 & \mathscr{B}_{t_*}[i] = \mathscr{T}_{t_*}[i]
% 		\end{cases}
% 	\end{equation}
% 	and set $W = \sum_{i \in V} X_i$.
% 	For each $i \in V$, decompose $W$ into $W = X_i + U_i + Z_i + W_i$ where
% 	\begin{align}
% 		U_i &= \sum_{j \in B_i} X_j, &
% 		Z_i &= \sum_{j \in C_i} X_j, &
% 		W_i &= \sum_{j \in D_i} X_j.
% 	\end{align}
	% and $B_i, C_i$, and $D_i$ are the vertex sets



	% We now define the quantities
	% \begin{align}
	% 	\lambda &= \sum_{i \in V} \expect\left[\frac{X_i}{X_i + U_i} \indicator[ X_i + U_i \geq 1] \right],\\
	% 	\mu_l &= \frac{1}{l \lambda} \sum_{i \in V} \expect\left[ X_i \indicator[X_i + U_i = l] \right], && l\geq 1,
	% 	% \label{eq:mu definition}
	% \end{align}
	% which will be the parameters of the approximating compound Poisson distribution to $W$. We also define
	% \begin{align}
	% 	\delta_1 &= \sum_{i \in V}  \sum_{k \geq 0} \prob[X_i = 1, U_i = k] \expect \left|\frac{\prob[X_i = 1, U_i = k|W_i]}{\prob[X_i = 1, U_i = k]} - 1 \right|, \label{eq:delta1 nD}\\ 
	% 	% \delta_2 &= \\
	% 	% \delta_3 &= \\
	% 	\delta_4 &= \sum_{i \in V} \left( \expect[X_i Z_i] + \expect[X_i] \expect[X_i + U_i + Z_i] \right), \label{eq:delta4 nD}
	% \end{align}
	% % which we will require to vanish as $n \rightarrow \infty$.
	% which we desire to be small for the compound Poisson approximation to be good.

	% The following theorem (reworked from \cite{Barbour2001-nh}) bounds the distance between the distributions of $W$ and the approximating compound Poisson.

	% \begin{theorem}[\cite{Barbour2001-nh}]
	% \label{thm: nd compound poisson approximation}
	% 	Let $W$, $\lambda$, $\vect{\mu}$, $\delta_1$ and $\delta_4$ be as defined above. Then
	% 	\begin{equation}
	% 		d_{\mathrm{TV}}(\mathscr{L}(W), \mathrm{CP}(\lambda, \vect{\mu})) \leq (\delta_1 + \delta_4)\euler^\lambda.
	% 	\end{equation}
	% \end{theorem}
	% As per the discussion proceeding Theorem \ref{thm: compound poisson approximation}, we obtain the following as a corollary to Theorem \ref{thm: nd compound poisson approximation}.
	\begin{corollary}
		\label{cor:nd bound distance prob[w=0] and e^-lambda}
		Let $T_n$ be the coupling time of the continuous-time heat-bath Glauber dynamics for the zero-field Ising model at inverse-temperature $\beta$ on the graph $G_n$ and let $\delta_1$, $\delta_4$ and $\lambda$ be as defined above. Then
		\begin{equation}
			\left|\prob\left[T_n \leq z + t_c(n)\right] - \euler^{-\lambda}\right| \leq (\delta_1 + \delta_4)\euler^\lambda.
			\label{eq:nd bound distance prob[w=0] and e^-lambda}
		\end{equation}
	\end{corollary}

\section{Proof of Theorem \ref{thm:Coupling Distribution on transitive graph}}
\label{sec:proof thm coupling nd}
	In this section we use Corollary \ref{cor:nd bound distance prob[w=0] and e^-lambda} to prove Theorem \ref{thm:Coupling Distribution on transitive graph} by bounding $\lambda$ and showing that $\delta_1$ and $\delta_4$ go to zero as $n \rightarrow \infty$. This is done in Lemmas \ref{lem:nDlambda}, \ref{lem:delta1 goes to 0 general}, and \ref{lem:delta4 goes to 0 general}. The proofs of these require some additional lemmas concerning properties of the update histories which have been deferred to Section \ref{sec:additional lemmas nd}.

	We begin by bounding $\lambda$. Note that bounding $\lambda$ is enough to show that there is a subsequence of graphs on which $\lambda$ converges as required by Theorem \ref{thm:Coupling Distribution on transitive graph}.
	\begin{lemma}
	\label{lem:nDlambda}
		Using the above set-up, 
		\begin{equation}
			\limsup_{n \rightarrow \infty} \lambda \leq \max(\euler^{-z}, \euler^{-(1 - \beta\Delta)z}),
			\label{eq:limsup lambda nD}
		\end{equation}
		and for small enough $\beta$ there exists a constant $C \in (0,1)$ such that
		\begin{equation}
			\liminf_{n \rightarrow \infty} \lambda \geq C \min(\euler^{-z}, \euler^{-(1 - \beta\Delta)z}).
			\label{eq:liminf lambda nD}
		\end{equation}
	\end{lemma}
	\begin{proof}
		Starting with the definition of $\lambda$, we have
		\begin{align}
			\lambda &= \sum_{i \in V} \expect\left[\frac{X_i}{X_i + U_i} \indicator[ X_i + U_i \geq 1] \right]\\
				&= \sum_{i = 1}^n \prob(X_i = 1) \expect \left[\left.\frac{1}{1 + U_i}\right| X_i = 1\right]\\
				&= n \, m_{t_*} \, \expect \left[\left.\frac{1}{1 + U_i}\right| X_i = 1\right],
		\end{align}
		where we have used that $X_i$ is zero-one, \eqref{eq:prob X_i = 1 is m_t*}, and the transitivity of the graph.
		Since $U_i$ is non-negative we have
		\begin{align}
			\expect \left[\frac{1}{1 + U_i}| X_i = 1\right] \leq 1
		\end{align}
		and so by Corollary \ref{cor:magnetization of t star}, $\lambda \leq n m_{t_*} \leq \max(\euler^{-z}, \euler^{-(1 - \beta\Delta)z})$, establishing \eqref{eq:limsup lambda nD}.

		By Jensen's inequality
		\begin{align}
			\expect \left[\frac{1}{1 + U_i}| X_i = 1\right] &\geq \frac{1}{\expect[1 + U_i | X_i = 1]}\\
				&= \frac{1}{1 + \expect[U_i | X_i = 1]},
		\end{align}
		so in order to find a lower bound for $\lambda$ we will find an upper bound to $\expect[U_i | X_i = 1]$. Now by Lemma \ref{lem:nd X_j given X_i}, there exists a $C_1>0$ such that for small enough $\beta$,
		\begin{align}
			\expect[U_i | X_i = 1] &= \sum_{j \in B_i} \prob[X_j = 1| X_i = 1]\\
				% &= \sum_{k=1}^{\lfloor b_n \rfloor} \sum_{|j - i| = k} \prob[X_j = 1| X_i = 1]\\
				% & \leq |B_i| m_{t_*} + C_1\sum_{k=1}^{\lfloor b_n \rfloor} \sum_{j: d(i,j) = k} \euler^{-k}\\
				&\leq |B_i|m_{t_*} + C_1 \sum_{k=1}^{\lfloor b_n \rfloor} P_n(k) \euler^{-k}\\
				&\leq |B_i|m_{t_*} + C_1 \sum_{k=1}^{\infty} P_n(k) \euler^{-k}.
		\end{align}
		By Corollary \ref{cor:magnetization of t star}, $m_{t_*} \leq \max(\euler^{-z}, \euler^{-(1 - \beta\Delta)z})/n$, and from \eqref{eq:set valid graphs}, $|B_i| \leq Q_n(\ln n) \leq Q_n (\ln^2 n) = o(n)$.
		So as $n \rightarrow \infty$, the first term vanishes, and from \eqref{eq:set valid graphs} we have that for some $C_2 > 0$,
		\begin{equation}
			 \limsup_{n \rightarrow \infty} \left(C_1 \sum_{k=1}^\infty P_n(k) \euler^{-k} \right)\leq C_2.
		\end{equation}
		% \begin{align}
		% 	\expect[U_i | X_i = 1] %&\leq |B_i|m_{t_*} + C_1 \sum_{k=1}^{\lfloor b_n \rfloor} P_n(k) \euler^{-k}\\
		% 	%&\leq |B_i|m_{t_*} + C_1 \sum_{k=1}^{\infty} P_n(k) \euler^{-k}\\
		% 	&\leq C_z|B_i|/n + C_2
		% \end{align}
		% for some $C_2 > 0$, 
		% As $n \rightarrow \infty$, the first term vanishes and we are left with 
		Hence
		\begin{align}
			\liminf_{n \rightarrow \infty} \lambda &\geq \frac{1}{1 + C_2} n m_{t_*}\\
				&\geq C \min(\euler^{-z}, \euler^{-(1 - \beta\Delta)z}) 
		\end{align}
		for some $C \in (0, 1)$, since $m_{t_*} \geq \frac{1}{n}\min(\euler^{-z}, \euler^{-(1 - \beta\Delta)z})$ by Corollary \ref{cor:magnetization of t star}.
	\end{proof}

	% \begin{lemma}
	% 	[LAMBDA CONVERGES]
	% \end{lemma}
	% \begin{proof}
		
	% \end{proof}
	\begin{lemma}
	\label{lem:delta1 goes to 0 general}
		Let $\delta_1$ be as defined in \eqref{eq:delta_1 definition}. Then at any inverserse temperature $\beta \geq 0$,
		\begin{equation}
			\lim_{n\rightarrow\infty} \delta_1 = 0.
		\end{equation}
	\end{lemma}
	\begin{proof}
		Starting with the definition of $\delta_1$, we have
		\begin{align}
			\delta_1 &= \sum_{i = 1}^n \sum_{k = 0}^{|B_i|} \prob[X_i = 1, U_i = k] \expect \left| \frac{\prob[X_i = 1, U_i = k|W_i]}{\prob[X_i = 1, U_i = k]} - 1 \right|\\
			&= n \sum_{k = 0}^{|B_i|} \expect \left|\prob[X_i = 1, U_i = k|W_i] - \prob[X_i = 1, U_i = k] \right|,%\\
			\label{eq:nD delta1 line 2}
			% &\leq n \sup_{W_i} \sum_{k = 0}^{2 b_n} \left| \prob[X_i = 1, U_i = k|W_i] - \prob[X_i = 1, U_i = k] \right|
		\end{align}
		by the transitivity of the graph.
		% Denote using
		% \begin{equation}
		% 	B(i, l) = \{j \in V : d(i, j) \leq l\}
		% \end{equation}
		% the set of points within distance $l$ of $i$ and
		% \begin{equation}
		% 	C_i^c = \{j : d(i,j) \leq (c_n + b_n)/2\}
		% \end{equation}
		% be the set of vertices within distance $(b_n + c_n)/2$ of $i$ and 
		Define the events
		\begin{equation}
			A_1 = \{\exists j \in B_i \cup \{i\}, \exists t \in [0, t_*] : \mathcal{H}_j(t) \nsubseteq  B(i, (c_n + b_n)/2)\}
		\end{equation}
		and
		\begin{equation}
			A_2 = \{\exists j \in D_i, \exists t \in [0, t_*] : \mathcal{H}_j(t) \cap B(i, (c_n + b_n)/2) \neq \emptyset\}
		\end{equation}
		as well as their intersection
		\begin{equation}
			A = A_1 \cap A_2.
		\end{equation}

		From Lemma \ref{lem:conditioning on A removes conditioning on W},
		\begin{equation}
			\prob[X_i = 1, U_i = j | A^\complement, W_i] = \prob[X_i = 1, U_i = j | A^\complement].
		\end{equation} 
		Continuing on from \eqref{eq:nD delta1 line 2}, we split the probabilities into
		\begin{align}
			\delta_1 &= n \sum_{k = 0}^{|B_i|} \expect \left| \prob[X_i = 1, U_i = k|W_i, A]\prob[A|W_i] - \prob[X_i = 1, U_i = k| A]\prob[A] + \vphantom{A^\complement} \vphantom{\frac{1}{2}}\right.\\
			&\hphantom{= .} \left.\prob(X_i = 1, U_i = k | A^\complement) (\prob[A^\complement|W_i] - \prob[A^\complement]) \vphantom{\frac{1}{2}}\right| \notag \\
			&\leq n (|B_i| + 1) \expect\left[ \prob[A|W_i] + \prob[A] + \left|\prob[A^\complement|W_i] - \prob[A^\complement]\right|\right]\\
			&= n (|B_i| + 1) \expect\left[ \prob[A|W_i] + \prob[A] + \left|1 - \prob[A|W_i] - (1 - \prob[A])\right|\right]\\
			&\leq n (|B_i| + 1) \expect\left[ \prob[A|W_i] + \prob[A] + \prob[A|W_i] + \prob[A])\right]\\
			&= 2 n (|B_i| + 1) \left(\expect[\prob[A|W_i]] + \prob[A]\right)\\
			&= 4 n (|B_i| + 1) \prob[A].
		\end{align}

		% For either $A_1$ or $A_2$ to hold, there must exists a history that spreads at least distance $(c_n - b_n) / 2$ away from its starting vertex. 
		% By a union bound
		For $A_1$ to hold, we must have a history $\hist_j$ that extends from a distance of no more than $b_n$ from $i$ to a distance of at least $(b_n + c_n)/2$ from $i$. That is, it must extend a distance of at least $(c_n - b_n)/2$. For $A_2$ to hold, we must have a history $\hist_j'$ that extends from a distance of at least $c_n$ from $i$ to a distance of no more than $(b_n + c_n)/2$ from $i$. That is, it must extend a distance of at least $(c_n - b_n)/2$. So for either $A_1$ or $A_2$ to hold, there must exist a history that spreads at least distance $(c_n - b_n) / 2$ away from its starting vertex. 
		By a union bound
		\begin{align}
			\prob[A] &\leq \prob[A_1 \cup A_2]\\
				&\leq \prob\left[\bigcup_{j \in V} \left\{\hist_j \nsubseteq B(j, (c_n - b_n)/2) \times [0, t_*] \right\} \right]\\
				&\leq \sum_{j \in V} \prob[\mathcal{H}_j \nsubseteq B(j, (c_n - b_n)/2) \times [0, t_*]]\\
				&= n \prob\left[\bigcup_{u \in [0, t_*]}\mathcal{H}_j(t_* - u) \nsubseteq B(j, (c_n - b_n)/2)\right].
		\end{align}
		Combining this with Lemma \ref{lem:prob history contained in ball} we get that
		\begin{align}
			\delta_1 &\leq 4 n^2 (|B_i| + 1) \exp(t_* \Delta^2 - \ln(\Delta)(c_n - b_n)/2)\\
			&\leq 4 \exp(\Delta^2 z) n^{2 + \Delta^2/(1 - \beta \Delta)}(|B_i| + 1) \exp(-\ln (\Delta) (c_n - b_n)/2)
		\end{align}
		using Corollary \ref{cor:bound t_c(n)}. Recalling our choices of $b_n = \ln(n)$ and $c_n = \ln(n)^2$, we have
		\begin{align}
			\delta_1 &\leq 4 \euler^{\Delta^2 z} n^{2 + \Delta^2/(1 - \beta \Delta)} Q_n(b_n) \euler^{-\frac{\ln(\Delta)}{2} c_n} \euler^{\frac{\ln(\Delta)}{2} b_n}\\
			&= 4 \euler^{\Delta^2 z} n^{2 + \Delta^2/(1 - \beta \Delta) + \log(\Delta)/2} Q_n(b_n) \euler^{-\frac{\ln(\Delta)}{2} \ln^2(n)}\\
			&\leq 4 \euler^{\Delta^2 z} n^{2 + \Delta^2/(1 - \beta \Delta) + \log(\Delta)/2} Q_n(\ln^2(n)) \euler^{-\frac{\ln(\Delta)}{2} \ln^2(n)},
		\end{align}
		which by \eqref{eq:set valid graphs} goes to $0$ as $n \rightarrow \infty$.

		% By a union bound,
		% \begin{align}
		% 	\delta_1 &\leq 4 n (|B_i| + 1) \sum_{j \in B_i \cup \{i\}} \sum_{k \in D_i} \prob[\mathcal{H}_{j} \cap \mathcal{H}_k \neq \emptyset].
		% \end{align}

		% [FIX THIS LEMMA]
		% % From Lemma \ref{lem:intersecting histories bound},
		% \begin{align}
		% 	\delta_1 &\leq 4 n^2 (|B_i| + 1)^2 \exp(2\alpha) \exp(-\lambda(c_n - b_n)).
		% \end{align}

		% On the torus $(\mathbb{Z} / L \mathbb{Z})^d$, $n = L^d$, $c_n = \log(L)^2$, $b_n = \log(L)$, and $|B_i| \leq C \log(L)^d$. So

		% \begin{equation}
		% 	\delta_1 \leq C L^{2d} \log(L)^{2d} \exp(-\lambda(\log(L)^2 - \log(L))
		% \end{equation}
		% which goes to $0$ as $L \rightarrow \infty$.
	\end{proof}

	\begin{lemma}
	\label{lem:delta4 goes to 0 general}
		Let $\delta_4$ be as defined in \eqref{eq:delta_4 definition}. Then for small enough $\beta$,
		\begin{equation}
			\lim_{n\rightarrow\infty} \delta_4 = 0.
		\end{equation}
	\end{lemma}
	\begin{proof}
		Starting with the definition of $\delta_4$, we have
		\begin{align}
			\delta_4 &= \sum_{i = 1}^n \left(\expect[X_i Z_i] + \expect[X_i]\expect[X_i + U_i + Z_i]\right)\\
				&= n \expect[X_i Z_i] + n m_{t_*}^2 \left(1 + |B_i| + |C_i|\right)\\
				&= n m_{t_*} \expect[Z_i | X_i = 1] + n m_{t_*}^2 \left(1 + |B_i| + |C_i|\right)\\
				&\leq C_z \expect[Z_i | X_i = 1] + \frac{C_z^2}{n} \left(1 + |B_i| + |C_i|\right)\\
				&\leq C_z \expect[Z_i | X_i = 1] + \frac{C_z^2}{n} \left(1 + 2 Q_n(\ln^2(n))\right),
				% &= n m_{t_*} \left(\sum_{j \in C_i} \prob[X_j = 1 | X_i = 1] + m_{t_*} \left(1 + |B_i| + |C_i|\right) \right)%\\
				% &\leq n m_{t_*} \left(\sum_{j \in C_i} \left(m_{t_*} + \prob[\mathcal{H}_{j} \cap \mathcal{H}_i \neq \emptyset | X_i = 1\right) + m_{t_*} \left(1 + |B_i| + |C_i|\right) \right)
		\end{align}
		where using Corollary \ref{cor:magnetization of t star}
		\begin{equation}
			C_z = \max(\euler^{-z}, \euler^{-(1 - \beta \Delta)z}).
		\end{equation}
		Now from \eqref{eq:set valid graphs}, the second term above vanishes as $n \rightarrow \infty$. So we turn our attention to the first term. From Lemma \ref{lem:nd X_j given X_i}, there exists a $C > 0$ such that for small enough $\beta$,
		\begin{align}
			\expect[Z_i | X_i = 1] &= \sum_{j \in C_i} \prob[X_j = 1 | X_i = 1]\\
				% &= \sum_{k = \lfloor b_n \rfloor + 1}^{\lfloor c_n \rfloor} \sum_{j: |i - j| = k} \prob[X_j = 1 | X_i = 1]\\
				&\leq \sum_{j \in C_i} \left[m_{t_*} + C \euler^{-d(i,j)}\right]\\
				&\leq \sum_{j \in C_i} \left[\frac{C_z}{n} + C \euler^{-b_n}\right]\\
				% &\leq |C_i| \left(m_{t_*} + C \euler^{-b_n}\right)\\
				&= |C_i| \left(\frac{C_z}{n} + \frac{C}{n}\right)\\
				&\leq Q_n(\ln^2(n)) \left(\frac{C_z}{n} + \frac{C}{n}\right),
		\end{align}
		where we have again used Corollary \ref{cor:magnetization of t star}. From \eqref{eq:set valid graphs}, we get that $\delta_4$ goes to zero as $n \rightarrow \infty$.
		% NEED TO PROVE LEMMA \ref{lem:intersecting histories bound given X_j = 1}
	\end{proof}

\section{Additional lemmas}
\label{sec:additional lemmas nd}

This section contains the statements and proofs for a number of lemmas concerning properties of the update histories that are required for the proofs in Section \ref{sec:proof thm coupling nd}.

\begin{lemma}
		There exists a constant $C > 0$ such that for sufficiently small $\beta$,
		\label{lem:nd X_j given X_i}
		\begin{equation}
			\prob[X_j = 1| X_i = 1] \leq m_{t_*} + C \euler^{-k},
		\end{equation}
		where $k = d(i,j)$ is the distance between vertices $i$ and $j$ on a transitive graph.
	\end{lemma}
	\begin{proof}
		There are two ways in which the update history of vertex $j$ can survive until time $0$. The update history can survive without intersecting with the update history of vertex $i$ or the update history of vertex $j$ can merge with the update history of vertex $i$ (whose survival we are conditioning on). %(Note that these events are not mutually exclusive as the update history of vertex $i+k$ could merge with the update history of vertex $i$ after it reaches time $0$.) 
		Breaking up the probability this way we have
		\begin{align}
			\prob[X_j = 1| X_i = 1] &= \prob[X_j = 1, \mathcal{H}_i \cap \mathcal{H}_j = \emptyset | X_i = 1] \notag\\
			&\phantom{=}+ \prob[X_j = 1, \mathcal{H}_i \cap \mathcal{H}_j \neq \emptyset | X_i = 1]\\
			&\leq \prob[X_j = 1, \mathcal{H}_i \cap \mathcal{H}_j = \emptyset | X_i = 1] + \prob[\mathcal{H}_i \cap \mathcal{H}_j \neq \emptyset | X_i = 1].
		\end{align}

		The result follows from Lemmas \ref{lem: prob X_j and no intersect given X_i = 1} and \ref{lem: prob intersect given X_i = 1}.
	\end{proof}

	
	% \begin{lemma}
	% 	\label{lem:intersecting histories bound}
	% 	\begin{equation}
	% 		\prob[\mathcal{H}_{j} \cap \mathcal{H}_k \neq \emptyset] \leq \exp(2\alpha) \exp(-\lambda |j - k|).
	% 	\end{equation}
	% \end{lemma}
	% \begin{proof}
	% 	The following is based on the proof used in Section 3.2 of \cite{Lubetzky2014-po}.

	% 	We first relax our histories to our alternative construction by observing that
	% 	\begin{equation}
	% 		\prob[\mathcal{H}_j \cap \mathcal{H}_k \neq \emptyset] \leq \prob[\hat{\mathcal{H}}_j \cap \hat{\mathcal{H}}_k \neq \emptyset].
	% 	\end{equation}

	% 	Let $W_s = |\hat{\mathcal{H}}_{\{j, k\}} (t_* - s)|$ and let $Y_s = \#\left\{ \left( (u,t), (v,t) \right) \in \hat{\mathcal{H}}_{\{j, k\}}: t \in [t_* - s, t_*) \right\}$ count the total number of spatial edges observed in the history by time $t_* - s$. For $\hat{\mathcal{H}}_j$ and $\hat{\mathcal{H}}_k$ to intersect at time $t_* - s$, there must be enough spatial edges for the histories to reach each other. That is, we require that
	% 	\begin{equation}
	% 		Y_s \geq |i - j|.
	% 	\end{equation}

	% 	Initially, $W_0 = 2$ and $Y_0 = 0$. Recall that an oblivious update of a vertex causes it to be removed from the history and that a non-oblivious update causes the history to branch out to its $\Delta$ neighbours. Oblivious updates occur at rate $\theta W_s$ and cause $W_s$ to decrease by $1$. Non-oblivious updates occur at rate $(1 - \theta)W_s$ and cause both $W_s$ and $Y_s$ to increase by no more than $\Delta$. Therefore we can create a coupled process $(\bar{W}_s, \bar{Y}_s)$ such that $\bar{W}_s \geq W_s$ and $\bar{Y}_s \geq Y_s$ in the following way. We start with $(\bar{W}_s, \bar{Y}_s) = (2, 0)$ and at rate $\theta \bar{W}_s$, $\bar{W}_s$ decreases by $1$, and at rate $(1 - \theta)\bar{W}_s$, both $\bar{W}_s$ and $\bar{Y}_s$ increase by $\Delta$.

	% 	Let $Q_s = \exp \left(\alpha \bar{W}_s + \lambda \bar{Y}_s \right)$ where $\alpha$ and $\lambda$ are some fixed constants yet to be determined. We will show that $Q_s$ is a supermartingale. Let $h$ be some small time-step. Then
	% 	\begin{align}
	% 		\expect\left[Q_{s_0 + h} - Q_{s_0}| Q_{s_0}\right] &= h \theta \bar{W}_{s_0} \left(\exp(\alpha (\bar{W}_{s_0} - 1) + \lambda \bar{Y}_{s_0}) - \exp(\alpha \bar{W}_{s_0} + \lambda \bar{Y}_{s_0})\right) + \\
	% 		&\hphantom{= .} h (1 - \theta) \bar{W}_{s_0} \left( \exp(\alpha (\bar{W}_{s_0} + \Delta) + \lambda (\bar{Y}_{s_0} + \Delta)) - \exp(\alpha \bar{W}_{s_0} + \lambda \bar{Y}_{s_0}) \right) + \notag\\
	% 		&\hphantom{= .} \mathcal{O}(h^2) \notag\\
	% 		&= \left(\theta (\euler^{-\alpha} - 1) + (1 - \theta)(\euler^{(\alpha + \lambda)\Delta} - 1)\right)h \bar{W}_{s_0} Q_{s_0} + \mathcal{O}(h^2).
	% 	\end{align}
	% 	Dividing through by $h$ and taking $h$ to $0$, we have
	% 	\begin{equation}
	% 		\left.\frac{d}{ds} \expect\left[Q_{s} | Q_{s_0}\right] \right|_{s = s_0} = \left(\theta (\euler^{-\alpha} - 1) + (1 - \theta)(\euler^{(\alpha + \lambda)\Delta} - 1)\right)\bar{W}_{s_0} Q_{s_0}
	% 	\end{equation}
	% 	which is negative for $\theta$ sufficiently close to $1$. Hence $Q_s$ is a supermartingale for sufficiently large $\theta$.

	% 	Define the stopping time 
	% 	\begin{equation}
	% 		\tau = \inf\{s: \bar{W}_s = 0\}.
	% 	\end{equation}
	% 	Since $(\bar{W}_s, \bar{Y}_s) \geq (W_s, Y_s)$, we have that $W_\tau = 0$ which corresponds to the event that both histories, $\mathcal{H}_j$ and $\mathcal{H}_k$, have terminated by time $\tau$. Hence
	% 	\begin{equation}
	% 		\prob[\hat{\mathcal{H}}_j \cap \hat{\mathcal{H}}_k \neq \emptyset] \leq \prob[Y_\tau \geq |j-k|].
	% 	\end{equation}
	% 	Now
	% 	\begin{equation}
	% 		\expect[\exp(\lambda Y_\tau)] \leq \expect[\exp(\lambda\bar{Y}_\tau)].
	% 	\end{equation}
	% 	and from optional stopping [FIGURE THIS OUT PROPERLY],
	% 	\begin{align}
	% 		\expect[\exp(\lambda \bar{Y}_\tau )] &\leq \expect[Q_{0}]\\
	% 		&= \exp(2\alpha).
	% 	\end{align}
	% 	Hence 
	% 	\begin{equation}
	% 		\prob[Y_\tau \geq |j-k|] \leq \exp(2\alpha) \exp(-\lambda |j - k|)
	% 	\end{equation}
	% 	and overall we have that
	% 	\begin{equation}
	% 		\prob[\mathcal{H}_j \cap \mathcal{H}_k \neq \emptyset] \leq \exp(2\alpha) \exp(-\lambda |j - k|).
	% 	\end{equation}
	% \end{proof}

	% \begin{lemma}
	% 	\label{lem:history escaping ball bound}
	% 	\begin{equation}
	% 		\prob[\mathcal{H}_{i} \not\subset B(i, l)] \leq \exp(\alpha) \exp(-\lambda l).
	% 	\end{equation}
	% \end{lemma}
	% \begin{proof}
	% The following is based on the proof used in Section 3.2 of \cite{Lubetzky2014-po}.

	% We first relax our histories to our alternative construction by observing that
	% \begin{equation}
	% 	\prob[\mathcal{H}_{i} \not\subset B(i, l)] \leq \prob[\hat{\mathcal{H}}_{i} \not\subset B(i, l)].
	% \end{equation}

	% Let $W_s = |\hat{\mathcal{H}}_i (t_* - s)|$ and let $Y_s = \#\left\{ \left( (u,t), (v,t) \right) \in \hat{\mathcal{H}}_i: t \in [t_* - s, t_*) \right\}$ count the total number of spatial edges observed in the history by time $t_* - s$. For $\hat{\mathcal{H}}_i$ to extend beyond $B(i, l)$, there must be enough spatial edges. That is, we require that
	% \begin{equation}
	% 	Y_s \geq l.
	% \end{equation}

	% Initially, $W_0 = 1$ and $Y_0 = 0$. Recall that an oblivious update of a vertex causes it to be removed from the history and that a non-oblivious update causes the history to branch out to its $\Delta$ neighbours. Oblivious updates occur at rate $\theta W_s$ and cause $W_s$ to decrease by $1$. Non-oblivious updates occur at rate $(1 - \theta)W_s$ and cause both $W_s$ and $Y_s$ to increase by no more than $\Delta$. Therefore we can create a coupled process $(\bar{W}_s, \bar{Y}_s)$ such that $\bar{W}_s \geq W_s$ and $\bar{Y}_s \geq Y_s$ in the following way. We start with $(\bar{W}_s, \bar{Y}_s) = (1, 0)$ and at rate $\theta \bar{W}_s$, $\bar{W}_s$ decreases by $1$, and at rate $(1 - \theta)\bar{W}_s$, both $\bar{W}_s$ and $\bar{Y}_s$ increase by $\Delta$.

	% Let $Q_s = \exp \left(\alpha \bar{W}_s + \lambda \bar{Y}_s \right)$ where $\alpha$ and $\lambda$ are some fixed constants yet to be determined. We will show that $Q_s$ is a supermartingale. Let $h$ be some small time-step. Then
	% \begin{align}
	% 	\expect\left[Q_{s_0 + h} - Q_{s_0}| Q_{s_0}\right] &= h \theta \bar{W}_{s_0} \left(\exp(\alpha (\bar{W}_{s_0} - 1) + \lambda \bar{Y}_{s_0}) - \exp(\alpha \bar{W}_{s_0} + \lambda \bar{Y}_{s_0})\right) + \\
	% 	&\hphantom{= .} h (1 - \theta) \bar{W}_{s_0} \left( \exp(\alpha (\bar{W}_{s_0} + \Delta) + \lambda (\bar{Y}_{s_0} + \Delta)) - \exp(\alpha \bar{W}_{s_0} + \lambda \bar{Y}_{s_0}) \right) + \notag\\
	% 	&\hphantom{= .} \mathcal{O}(h^2) \notag\\
	% 	&= \left(\theta (\euler^{-\alpha} - 1) + (1 - \theta)(\euler^{(\alpha + \lambda)\Delta} - 1)\right)h \bar{W}_{s_0} Q_{s_0} + \mathcal{O}(h^2).
	% \end{align}
	% Dividing through by $h$ and taking $h$ to $0$, we have
	% \begin{equation}
	% 	\left.\frac{d}{ds} \expect\left[Q_{s} | Q_{s_0}\right] \right|_{s = s_0} = \left(\theta (\euler^{-\alpha} - 1) + (1 - \theta)(\euler^{(\alpha + \lambda)\Delta} - 1)\right)\bar{W}_{s_0} Q_{s_0}
	% \end{equation}
	% which is negative for $\theta$ sufficiently close to $1$. Hence $Q_s$ is a supermartingale for sufficiently large $\theta$.

	% Define the stopping time 
	% \begin{equation}
	% 	\tau = \inf\{s: \bar{W}_s = 0\}.
	% \end{equation}
	% Since $(\bar{W}_s, \bar{Y}_s) \geq (W_s, Y_s)$, we have that $W_\tau = 0$ which corresponds to the event the history, $\hat{\mathcal{H}}_i$, has terminated by time $\tau$. Hence
	% \begin{equation}
	% 	\prob[\hat{\mathcal{H}}_{i} \not\subset B(i, l)] \leq \prob[Y_\tau \geq l].
	% \end{equation}
	% Now
	% \begin{equation}
	% 	\expect[\exp(\lambda Y_\tau)] \leq \expect[\exp(\lambda\bar{Y}_\tau)].
	% \end{equation}
	% and from optional stopping [FIGURE THIS OUT PROPERLY],
	% \begin{align}
	% 	\expect[\exp(\lambda \bar{Y}_\tau )] &\leq \expect[Q_{0}]\\
	% 	&= \exp(\alpha).
	% \end{align}
	% Hence 
	% \begin{equation}
	% 	\prob[Y_\tau \geq l] \leq \exp(\alpha) \exp(-\lambda l)
	% \end{equation}
	% and overall we have that
	% \begin{equation}
	% 	\prob[\mathcal{H}_i \not\subset B(i, l)] \leq \exp(\alpha) \exp(-\lambda l).
	% \end{equation}
	% \end{proof}
	% \begin{lemma}
	% 	\label{lem:intersecting histories bound given X_j = 1}
	% 	\begin{equation}
	% 		\prob[\mathcal{H}_{i} \cap \mathcal{H}_j \neq \emptyset | X_i = 1] \leq  2 \exp(z \Delta^2) \max(\euler^{-z}, \euler^{-(1 - \beta\Delta)z})\Delta^{-k/2} n^{\Delta^2 + 1}
	% 	\end{equation}
	% 	where $k = |i - j|$.
	% \end{lemma}
	% \begin{proof}
	% 	\begin{align}
	% 		\prob[\mathcal{H}_{i} \cap \mathcal{H}_j \neq \emptyset | X_i = 1]  &\leq \frac{\prob[\mathcal{H}_{i} \cap \mathcal{H}_j \neq \emptyset] }{\prob[X_i = 1]}\\
	% 		&\leq n \max(\euler^{-z}, \euler^{-(1 - \beta\Delta)z}) \prob[\mathcal{H}_{i} \cap \mathcal{H}_j \neq \emptyset].
	% 	\end{align}
	% 	Now letting $k = |i - j|$ denote the distance between $i$ and $j$,
	% 	\begin{equation}
	% 		\prob[\mathcal{H}_{i} \cap \mathcal{H}_j \neq \emptyset] \leq 2 \prob[\mathcal{H}_i \nsubseteq B(i, k/2) ]
	% 	\end{equation}
	% 	since in order for $\mathcal{H}_i$ and $\mathcal{H}_j$ to intersect, at least one of them must extend beyond halfway the distance between $i$ and $j$. From Lemma \ref{cor:prob update function time 0 in B(i, l)},
	% 	\begin{align}
	% 		\prob[\mathcal{H}_{i} \cap \mathcal{H}_j \neq \emptyset | X_i = 1] &\leq 2 \exp(z \Delta^2) \max(\euler^{-z}, \euler^{-(1 - \beta\Delta)z})\Delta^{-k/2} n^{\Delta^2 + 1}
	% 	\end{align}
	% \end{proof}

	% \begin{lemma}
	% 	\label{lem:number vertices at distance k}
	% 	The number of vertices at distance $k$ on a degree $\Delta$ transitive graph is
	% 	\begin{equation}
	% 		P(k) = ???
	% 	\end{equation}
	% \end{lemma}
	% \begin{lemma}
	% 	\label{lem:size B_i}
	% 	The size of $B_i$ is bounded by 
	% 	\begin{equation}
	% 		|B_i| \leq b_n P(b_n)
	% 	\end{equation}
	% \end{lemma}
	% \begin{lemma}
	% 	\label{lem:size C_i}
	% 	The size of $C_i$ is bounded by 
	% 	\begin{equation}
	% 		|C_i| \leq c_nP(c_n)
	% 	\end{equation}
	% \end{lemma}
	% \begin{lemma}
	% 	On the $d$ dimensional lattice,
	% 	\begin{equation}
	% 		\sum_{k = b_n}^\infty \prob[\mathcal{H}_{i} \cap \mathcal{H}_j \neq \emptyset | X_i = 1] \leq 
	% 	\end{equation}
	% \end{lemma}
	% \begin{proof}
	% 	On the $d$ dimensional torus, the number of vertices at distance $k$ is no more than something like $k^d$ [NEED FIGURE THIS OUT AND PROVE IT].
	% 	\begin{align}
	% 		\sum_{j \in C_i} \prob[\mathcal{H}_{i} \cap \mathcal{H}_j \neq \emptyset | X_i = 1] &\leq \sum_{k = b_n}^\infty \sum_{j, |j - i| = k} \prob[\mathcal{H}_{i} \cap \mathcal{H}_j \neq \emptyset | X_i = 1] \\
	% 		&\leq \sum_{k = b_n}^\infty k^d \prob[\mathcal{H}_{i} \cap \mathcal{H}_j \neq \emptyset | X_i = 1]
	% 	\end{align}
	% 	From Lemma \ref{lem:intersecting histories bound given X_j = 1},
	% 	\begin{align}
	% 		\sum_{j \in C_i} \prob[\mathcal{H}_{i} \cap \mathcal{H}_j \neq \emptyset | X_i = 1] &\leq C(z, \Delta) n^{\Delta^2 + 1} \sum_{k = b_n}^\infty k^d \Delta^{-k/2} \\
	% 		&\leq C(z, \Delta) n^{\Delta^2 + 1} \sum_{k = b_n}^\infty k^d \exp(-\ln(\Delta) k/2)\\
	% 		&\approxeq C(z, \Delta) n^{\Delta^2 + 1} b_n^d \exp(-\ln(\Delta)b_n/2)
	% 	\end{align}
	% 	which goes to 0 for large enough $b_n$ [FIGURE OUT HOW BIG].
	% \end{proof}

	% \begin{lemma}
	% 	\label{lem:intersecting histories bound given X_j = 1}
	% 	\begin{equation}
	% 		\prob[\mathcal{H}_{j} \cap \mathcal{H}_k \neq \emptyset | X_j = 1] \leq 
	% 	\end{equation}
	% \end{lemma}
	% \begin{proof}
	% The proof here is similar to that of Lemma \ref{lem:intersecting histories bound}. We first relax to our alternative histories
	% \begin{equation}
	% 	\prob[\mathcal{H}_{j} \cap \mathcal{H}_k \neq \emptyset | X_j = 1] \leq \prob[\hat{\mathcal{H}}_{j} \cap \hat{\mathcal{H}}_k \neq \emptyset | X_j = 1].
	% \end{equation}
	% By symmetry, 
	% \begin{equation}
	% 	\prob[\hat{\mathcal{H}}_{j} \cap \hat{\mathcal{H}}_k \neq \emptyset | X_j = 1] = \prob[\hat{\mathcal{H}}_{j} \cap \hat{\mathcal{H}}_k \neq \emptyset | X_k = 1]
	% \end{equation}
	% and also
	% \begin{equation}
	% 	\prob[X_j = 1] = \prob[X_k = 1].
	% \end{equation}
	% So 
	% \begin{equation}
	% 	\prob[\hat{\mathcal{H}}_{j} \cap \hat{\mathcal{H}}_k \neq \emptyset | X_j = 1] \leq 2 \prob[\hat{\mathcal{H}}_{j} \cap \hat{\mathcal{H}}_k \neq \emptyset | \{X_j = 1\} \cup \{X_k = 1\}].
	% \end{equation}

	% We now define $W_s$ and $Y_s$ as before, except now conditioned on the event $\{X_j = 1\} \cup \{X_k = 1\}$. The effect of this conditioning is to forbid updates that reduce $W_s$ to $0$. 

	% HANDWAVING: If we stop when $W_s = 1$ then we can ignore the conditioning???

	% Define the stopping time 
	% \begin{equation}
	% 	\tau = \inf\{s: \bar{W}_s = 1\}.
	% \end{equation}

	% \end{proof}

	Our next lemma is based on \cite[Lemma 2.1]{Lubetzky2016-wd} and bounds the speed at which the histories can spread through the graph.
	\begin{lemma}
	\label{lem:prob history contained in ball}
		On any graph, the probability that the history of vertex $i$ escapes a ball of radius $l$ in time $s$ is bounded by
		\begin{equation}
			\prob\left[\bigcup_{u \in [0, s]}\mathcal{H}_i(t_* - u) \nsubseteq B(i, l)\right] \leq \exp\left(s \Delta^2 - l \ln \Delta \right).
		\end{equation}
	\end{lemma}
	\begin{proof}
		Let $\mathcal{W} = \{\vect{w} = (w_1, w_2, \dots, w_l) : w_1 = i, w_{k-1} \sim w_k\}$ be the set of length $l$ sequences of adjacent vertices starting at vertex $i$. That is, $\mathcal{W}$ is the set of all walks of length $l-1$ on $G_n$ that start at $i$. If $\mathcal{H}_i$ contains any vertex outside $B(i, l)$ at a time $u \in [t_* - s, t_*]$ then there must be some sequence $\vect{w} \in \mathcal{W}$ such that each $w_k$ was updated in order along the interval $[t_* - s, t_*]$. For any particular sequence $\vect{w}$, let $M_\vect{w}$ be the event that each $w_k$ was updated at some time $t_k$ such that $t_* > t_k > t_* - s$ and $t_{k-1} > t_k$. 

		For $M_\vect{w}$ to occur, we require that the $l$ independent rate 1 Poisson clocks associated with each $w_k$ ring in a particular order along $[t_* - s, t_*]$. The probability of this is equal to the probability that a single rate 1 Poisson clock rings at least $l$ times in $[0, s]$, or equivalently, a rate $s$ Poisson clock rings at least $l$ times in $[0, 1]$. So we have
		\begin{equation}
			\prob[M_\vect{w}] = \prob[\mathrm{Po}(s) \geq l],
		\end{equation}
		where $\mathrm{Po}(s)$ is Poisson with rate $s$. By a union bound over $\mathcal{W}$,
		\begin{align}
			\prob\left[\bigcup_{u \in [0, s]}\mathcal{H}_i(t_* - u) \nsubseteq B(i, l)\right] &\leq \prob\left[\bigcup_{\vect{w} \in \mathcal{W}} M_\vect{w}\right]\\
			&\leq \Delta^{l-1} \prob[\mathrm{Po}(s) \geq l].
		\end{align}
		The moment generating function of a Poisson random variable with rate $s$ is
		\begin{equation}
			M(t) = \exp\left(s\left(\euler^t - 1\right)\right).
		\end{equation}
		Using a Chernoff bound, we have for every $t > 0$,
		\begin{equation}
			\prob[\mathrm{Po}(s) \geq l] \leq \exp\left(s\left(\euler^t - 1\right) - t l\right).
		\end{equation}
		Overall we have
		\begin{align}
			\prob\left[\bigcup_{u \in [0, s]}\mathcal{H}_i(t_* - u) \nsubseteq B(i, l)\right] &\leq \Delta^{l-1} \exp\left(s\left(\euler^t - 1\right) - t l\right)\\
			&\leq \exp\left(s\left(\euler^t - 1\right) + l (\ln \Delta - t) \right).
		\end{align}
		Choosing $t = 2 \ln \Delta$,
		\begin{align}
			\prob\left[\bigcup_{u \in [0, s]}\mathcal{H}_i(t_* - u) \nsubseteq B(i, l)\right] &\leq \exp\left(s\left(\Delta^2 - 1\right) - l \ln \Delta \right)\\
			&\leq \exp\left(s \Delta^2 - l \ln \Delta \right).
		\end{align}
	\end{proof}

	% \begin{corollary}
	% \label{cor:prob update function time 0 in B(i, l)}
	% 	\begin{equation}
	% 		\prob[\mathcal{H}_i(0) \nsubseteq B(i,l)] \leq \Delta^{-l} n^{\Delta^2} \exp(z \Delta^2)
	% 	\end{equation}
	% 	For $l \geq \log(n)(\Delta^2 + 1)/\log(\Delta)$
	% 	\begin{equation}
	% 			\prob[\mathcal{H}_i(0) \nsubseteq B(i,l)] \leq \frac{\exp(z \Delta^2)}{n}
	% 	\end{equation}
	% \end{corollary}


	%--------------------------------------------------------------------------
	% COULD USE THIS BIT MAYBE BUT NEED TO PROVE SOME STUFF TO GET IT TO WORK
	%--------------------------------------------------------------------------

	% [IF WE CAN SOMEHOW SHOW THAT $\prob[\hist_i \cap \hist_j \neq \emptyset|X_1 =1] \leq \prob[\bar{\hist}_i \cap\hist_j]$ THEN THIS GIVES US A BIGGER RANGE OF TEMPERATURES.]
	% \begin{lemma}
	% 	blah
	% \end{lemma}
	% \begin{proof}
	% 	Write $\bar{\mathcal{H}}_i$ for the history at $i$ but it only has branching updates at rate 1. Write $d_j$ for the time of death of history $j$. Let $k = |i-j|$.
	% 	\begin{align}
	% 		\prob(\bar{\mathcal{H}}_i \cap \mathcal{H}_j) &= \prob(\bar{\mathcal{H}}_i \cap \mathcal{H}_j | d_j < c(k)) \prob(d_j < c(k)) \notag \\
	% 		&\phantom{=}+ \prob(\bar{\mathcal{H}}_i \cap \mathcal{H}_j | d_j \geq c(k)) \prob(d_j \geq c(k))\\
	% 		&\leq \prob(\bar{\mathcal{H}}_i \cap \mathcal{H}_j | d_j < c(k)) + \prob(d_j \geq c(k))\\
	% 		&\leq \prob(\bar{\mathcal{H}}_i \cap \mathcal{H}_j | d_j < c(k)) + \euler^{-(1 - \beta \Delta) c(k)}\\
	% 		&\leq 2\prob(\bar{\mathcal{H}}_i(t_* - c(k)) \nsubseteq B(i, k/2)) + \euler^{-(1 - \beta \Delta) c(k)}\\
	% 		&\leq \exp\left(c(k) \Delta^2 - k/2 \ln \Delta \right) + \euler^{-(1 - \beta \Delta) c(k)}\\
	% 	\end{align}
	% 	Choosing $c(k) = \ln(k)^2$ works so long as $\beta \Delta < 1$.
	% \end{proof}

	%--------------------------------------------------------------------------
	%--------------------------------------------------------------------------




	% \begin{lemma}
	% 	Let $i$ and $j$ be the indices of two vertices at distance $k$. Then
	% 	\begin{equation}
	% 		\prob[X_j = 1 | X_i = 1] \leq
	% 	\end{equation}
	% \end{lemma}
	% \begin{proof}
	% 	Let $B(i, l)$ denote the set of vertices within distance $l$ of vertex $i$. Let $A$ be the event
	% 	\begin{equation}
	% 		\{\mathcal{H}_i \subset B(i,k/2), \mathcal{H}_j \subset B(j, k/2)\}.
	% 	\end{equation}
	% 	By the law of total probability,
	% 	\begin{align}
	% 		\prob[X_j = 1 | X_i = 1] &= \prob[X_j = 1 | X_i = 1, A] \prob[A|X_i = 1] \notag \\ 
	% 		&\phantom{=}  +\prob\left[X_j = 1 | X_i = 1, A^\complement\right] \prob\left[A^\complement|X_i = 1\right].
	% 	\end{align}
	% 	If we are conditioning on $A$, then the events $\{X_j = 1\}$ and $\{X_i = 1\}$ depend only on the update sequences within $B(j, k/2)$ and $B(i, k/2)$ respectively. This means that
	% 	\begin{equation}
	% 		\prob[X_j = 1 | X_i = 1, A] = \prob[X_j = 1 | A].
	% 	\end{equation}
	% 	So
	% 	\begin{align}
	% 		\prob[X_j = 1 | X_i = 1] &= \prob[X_j = 1 | A] \prob[A|X_i = 1] \notag \\ 
	% 		&\phantom{=}  +\prob\left[X_j = 1 | X_i = 1, A^\complement\right] \prob\left[A^\complement|X_i = 1\right]\\
	% 		&\leq \prob[X_j = 1 | A] + \prob\left[A^\complement | X_i = 1\right]
	% 	\end{align}
	% \end{proof}
	% \begin{lemma}
	% 	\begin{equation}
	% 		\prob[X_j = 1 | A] \leq 
	% 	\end{equation}
	% \end{lemma}
	% \begin{proof}
	% 	\begin{align}
	% 		\prob[X_j = 1 | A] &= \prob[X_j = 1 | \mathcal{H}_i \subset B(i, k/2), \mathcal{H}_j \subset B(j, k/2)]%\\
	% 		% &\leq \prob[X_j = 1]/\prob[\mathcal{H}_i \subset B(i, k/2), \mathcal{H}_j \subset B(j, k/2)]
	% 	\end{align}
	% 	The events $\{\mathcal{H}_i \subset B(i, k/2)\}$ and $\{\mathcal{H}_j \subset B(j, k/2)\}$ depend only the update sequence strictly within $B(i, k/2)$ and $B(j, k/2)$ respectively. When we condition on $\{\mathcal{H}_j \subset B(j, k/2)\}$ the event $\{X_j = 1\}$ depends only on the update sequence strictly within $B(j, k/2)$. So
	% 	\begin{align}
	% 		\prob[X_j = 1 | A] &= \prob[X_j = 1 | \mathcal{H}_j \subset B(j, k/2)]\\
	% 			&\leq m_{t_*} \prob[\mathcal{H}_j \subset B(j, k/2)]^{-1}\\
	% 			&= \frac{m_{t_*}}{1 - \prob[\mathcal{H}_j \not\subset B(j, k/2)]}.
	% 	\end{align}
	% 	From Lemma \ref{lem:history escaping ball bound}, 
	% 	\begin{align}
	% 		\prob[X_j = 1 | A] &\leq \frac{m_{t_*}}{1 - \exp(\alpha -\lambda k/2)}\\
	% 		&\leq \frac{m_{t_*}}{1 - \exp(\alpha - \lambda/2)}
	% 	\end{align}
	% 	since $k \geq 1$.

	% 	% So these events are independent. Continuing
	% 	% \begin{align}
	% 	% 	\prob[X_j = 1 | A] &\leq m_{t_*}^{-1} \prob[\mathcal{H}_i \subset B(i, k/2)]^{-2}\\
	% 	% 	&\sim \frac{1}{n} (1 - n^{\Delta^2} \Delta^{-k/2})^{-2}
	% 	% \end{align}
	% 	% blergh
	% \end{proof}
	% \begin{lemma}
	% 	\begin{equation}
	% 		\prob[A^\complement | X_i = 1] \leq 
	% 	\end{equation}
	% \end{lemma}
	% \begin{proof}
	% 	For ease of notation, let $A_i$ be the event
	% 	\begin{equation}
	% 		A_i = \{\mathcal{H}_i \subset B(j, k/2)\}.
	% 	\end{equation}
	% 	Obviously, 
	% 	\begin{equation}
	% 		A = A_i \cap A_j.
	% 	\end{equation}
	% 	Now
	% 	\begin{align}
	% 		\prob\left[A^\complement | X_i = 1\right] &\leq \prob\left[A_i^\complement | X_i = 1\right] + \prob\left[A_j^\complement | X_i = 1\right]\\
	% 		&= \prob\left[A_i^\complement | X_i = 1\right] + \prob\left[A_j^\complement | X_i = 1, A_i^\complement \right] \prob\left[A_i^\complement | X_i = 1\right] \notag \\
	% 		&\phantom{=} + \prob\left[A_j^\complement | X_i = 1, A_i \right] \prob\left[A_i | X_i = 1\right]\\
	% 		&\leq 2\prob\left[A_i^\complement | X_i = 1\right]+ \prob\left[A_j^\complement | X_i = 1, A_i \right].\\
	% 	\end{align}
	% 	If we condition on $A_i$ then the event $X_i = 1$ depends only on the update sequence strictly inside $B(i, k/2)$. Also note that the event $A_j$ depends only on the update sequence stricly inside $B(j, k/2)$. These update sequences are independent as they do not share any vertices and so
	% 	\begin{equation}
	% 		\prob\left[A_j^\complement | X_i = 1, A_i \right] = \prob\left[A_j^\complement \right]
	% 	\end{equation}
	% 	and we have
	% 	\begin{equation}
	% 		\prob\left[A^\complement | X_i = 1\right] \leq 2\prob\left[A_i^\complement | X_i = 1\right]+ \prob\left[A_j^\complement \right].
	% 	\end{equation}
	% \end{proof}
	% \begin{lemma}
	% 	\begin{equation}
	% 		\prob\left[\mathcal{H}_i \not\subset B(i, l) | X_i = 1\right] \leq
	% 	\end{equation}
	% \end{lemma}
	% \begin{proof}
	% 	\begin{align}
	% 		\prob\left[\mathcal{H}_i \not\subset B(i, l) | X_i = 1\right] &= m_{t_*}^{-1} \prob\left[\mathcal{H}_i \not\subset B(i, l), X_i = 1\right]\\
	% 		&\leq m_{t_*}^{-1} \prob\left[\chi(\mathcal{H}_i) \geq l, X_i = 1\right]\\
	% 		&\leq m_{t_*}^{-1} \prob\left[\chi(\mathcal{H}_i) \geq l, Z(\mathcal{H}_i) \geq t_*\right]\\
	% 		&= m_{t_*}^{-1} \expect\left[\indicator_{\chi(\mathcal{H}_i) \geq l} \indicator_{Z(\mathcal{H}_i) \geq t_*} \right]\\
	% 		&\leq m_{t_*}^{-1} \expect\left[\indicator_{\chi(\mathcal{H}_i) \geq l} \indicator_{Z(\mathcal{H}_i) \geq \ln(n) + z} \right]\\
	% 		&\leq m_{t_*}^{-1} \expect\left[\exp(\chi(\mathcal{H}_i) - l) \exp(Z(\mathcal{H}_i) - \ln(n) - z) \right]\\
	% 		&= m_{t_*}^{-1} \exp(-l) \exp(-z - \ln(n)) \expect\left[\exp(\chi(\mathcal{H}_i) + Z(\mathcal{H}_i)) \right]\\
	% 		&= \max(1, \exp(- \beta \Delta z)) \exp(-l) \expect\left[\exp(\chi(\mathcal{H}_i) + Z(\mathcal{H}_i)) \right]\\
	% 		&\leq C_z \euler^{-l} BLAH
	% 	\end{align}
	% \end{proof}

	\begin{lemma}
		\label{lem: prob X_j and no intersect given X_i = 1}
		Let $i$ and $j$ be two vertices on a vertex transitive graph. Then
		\begin{equation}
			\prob[X_j = 1, \hist_i \cap\hist_j = \emptyset|X_i = 1] \leq m_{t_*}.
		\end{equation}
	\end{lemma}
	\begin{proof}
		To begin
		\begin{align}
			\prob[X_j = 1, \hist_i \cap\hist_j = \emptyset|X_i = 1] &= \prob[X_i = 1]^{-1}\prob[X_i = 1, X_j = 1, \hist_i \cap\hist_j = \emptyset]\\
			&= \frac{\prob[\hist_i(0) \neq \emptyset, \hist_j(0) \neq \emptyset, \hist_i \cap \hist_j = \emptyset]}{\prob[\hist_i(0) \neq \emptyset]}.
		\end{align}
		By Lemma \ref{lem:prob Xj Xi and no intersect}, 
		\begin{equation}
			\prob[\hist_i(0) \neq \emptyset, \hist_j(0) \neq \emptyset, \hist_i \cap \hist_j = \emptyset] \leq \prob[\hist_i(0) \neq \emptyset]^2
		\end{equation}
		and so
		\begin{align}
			\prob[X_j = 1, \hist_i \cap\hist_j = \emptyset|X_i = 1] &\leq \prob[\hist_i(0) \neq \emptyset]\\
			&= m_{t_*}.
		\end{align}
	\end{proof}

	% \begin{lemma}
	% 	\label{lem: prob X_j and no intersect given X_i = 1}
	% 	There exists an $\alpha$ such that for $\beta$ sufficiently small
	% 	\begin{equation}
	% 		\prob[X_j = 1, \mathcal{H}_i \cap \mathcal{H}_j = \emptyset| X_i = 1] \leq C_{z, \epsilon} \exp(2 \alpha) n^{-\epsilon}.
	% 	\end{equation}
	% \end{lemma}
	% \begin{proof}
	% 	\begin{align}
	% 		\prob[X_j = 1, \hist_i \cap \hist_j = \emptyset| X_i = 1] &= m_{t_*}^{-1} \prob[X_j = 1, X_i = 1, \hist_i \cap \hist_j = \emptyset]\\
	% 		&\leq m_{t_*}^{-1} \prob\left[\frac{1}{2} \mathcal{L}(\hist_i \cup \hist_j) \geq t_*\right]\\
	% 		&= m_{t_*}^{-1} \prob\left[\frac{1+\epsilon}{2} \mathcal{L}(\hist_i \cup \hist_j) \geq (1+\epsilon) t_*\right]
	% 	\end{align}
	% 	for any $0 < \epsilon < 1$. Continuing
	% 	\begin{align}
	% 		\prob[X_j = 1, \hist_i \cap \hist_j = \emptyset| X_i = 1] &\leq m_{t_*}^{-1} \expect\left[\indicator_{(1+\epsilon)\mathcal{L}(\hist_i \cup \hist_j)/2 \geq (1+\epsilon)t_*}\right]\\
	% 		&= m_{t_*}^{-1} \expect\left[\exp((1+\epsilon)\mathcal{L}(\hist_i \cup \hist_j)/2 - (1+\epsilon)t_*)\right]\\
	% 		&= m_{t_*}^{-1} \exp(-(1+\epsilon)t_*) \expect\left[\exp\left(\frac{1+\epsilon}{2}\mathcal{L}(\hist_i \cup \hist_j\right)\right].
	% 	\end{align}
	% 	Since $t_* \geq \ln(n) + z$, and $m_{t_*} \geq \min(\euler^{-z}, \euler^{-(1 - \beta\Delta)z})/n$,
	% 	\begin{align}
	% 		\prob[X_j = 1, \hist_i \cap \hist_j = \emptyset| X_i = 1] &\leq C_{z, \epsilon} n^{-\epsilon} \expect\left[\exp\left(\frac{1+\epsilon}{2}\mathcal{L}(\hist_i \cup \hist_j\right)\right]
	% 	\end{align}
	% 	where 
	% 	\begin{equation}
	% 		C_{z, \epsilon} = \exp(-\epsilon z) \max(1, \euler^{\beta\Delta z}).
	% 	\end{equation}
	% 	From Lemma \ref{lem:full submartingale thing}, there exists an $\alpha$ such that for $\beta$ sufficiently small
	% 	\begin{equation}
	% 		\prob[X_j = 1, \hist_i \cap \hist_j = \emptyset| X_i = 1] \leq C_{z, \epsilon} \exp(2 \alpha) n^{-\epsilon}.
	% 	\end{equation}
	% \end{proof}
	The final two lemmas use two quantities, $\chi(\hist_A)$ and $\mathcal{L}(\hist_A)$, which in some sense measure the horizontal and vertical size of $\hist_A$ respectively. 
	Define
	\begin{equation}
		\chi(\mathcal{H}_i) = \#\left\{ \left( (u,t), (v,t) \right) \in \mathcal{H}_i \right\},
	\end{equation}
	which counts the total number of spatial edges in $\mathcal{H}_i$ and define
	\begin{equation}
		\mathcal{L}(\mathcal{H}_i) = \sum_{i \in V} \int_{0}^{t_*} \indicator_{(i, t) \in \mathcal{H}_i} \intd t,
	\end{equation}
	which is the sum of the lengths of all the temporal edges in $\hist_i$. 

	The following lemma contains some similarities to the proof of \cite[Lemma 2.1]{Lubetzky2015-po}. Indeed, the quantity 
	\begin{equation}
		\prob[\left\{\mathcal{H}_i \cap \mathcal{H}_j \neq \emptyset\right\} \cap \left\{\mathcal{H}_i(0) \cup \mathcal{H}_j(0) \neq \emptyset\right\}],
	\end{equation}
	which appears in \eqref{eq:prob in red star} below, is equivalent to the  expression
	\begin{equation}
		\prob[A \in \mathrm{RED}_A^*]
	\end{equation}
	when $A = \{i, j\}$ using the notation of that paper. We use their method to bound this probability, but add some extra steps for clarity.
	
	% [UP TO HERE WITH TIM'S COMMENTS] 

	\begin{lemma}
		\label{lem: prob intersect given X_i = 1}
		Let $i$ and $j$ be two vertices separated by distance $k$ on a transitive graph $G = (V,E)$ with fixed degree $\Delta$. Then there exists a $C$ such that for sufficiently small $\beta$,
		\begin{equation}
			\prob[\mathcal{H}_i \cap \mathcal{H}_j \neq \emptyset | X_i = 1] \leq C\euler^{-k}.
		\end{equation}
	\end{lemma}
	\begin{proof}
		Firstly,
		\begin{align}
			\prob[\mathcal{H}_i \cap \mathcal{H}_j \neq \emptyset | X_i = 1] &= \frac{\prob[\left\{\mathcal{H}_i \cap \mathcal{H}_j \neq \emptyset\right\} \cap \{X_i = 1\}]}{\prob[X_i = 1]}\\
			&\leq m_{t_*}^{-1} \prob[\left\{\mathcal{H}_i \cap \mathcal{H}_j \neq \emptyset\right\} \cap \{X_i = 1\}]\\
			&\leq  m_{t_*}^{-1} \prob[\left\{\mathcal{H}_i \cap \mathcal{H}_j \neq \emptyset\right\} \cap \left(\{X_i = 1\} \cup \{X_j = 1\}\right)]\\
			&= m_{t_*}^{-1} \prob[\left\{\mathcal{H}_i \cap \mathcal{H}_j \neq \emptyset\right\} \cap \left\{\mathcal{H}_i(0) \cup \mathcal{H}_j(0) \neq \emptyset\right\}],
			\label{eq:prob in red star}
		\end{align}
		since
		\begin{equation}
			\left\{X_i = 1\right\} = \left\{\mathcal{H}_i(0) \neq \emptyset\right\}.
		\end{equation}
		% then
		% \begin{align}
		% 	\prob[\mathcal{H}_i \cap \mathcal{H}_j \neq \emptyset | X_i = 1] &\leq m_{t_*}^{-1}\prob\left[\left\{\hat{\mathcal{H}}_i \cap \hat{\mathcal{H}}_j \neq \emptyset \right\} \cap \left\{\mathcal{H}_i(0) \neq \emptyset\right\}\right],\\
		% 	&\leq m_{t_*}^{-1}\prob\left[\left\{\hat{\mathcal{H}}_i \cap \hat{\mathcal{H}}_j \neq \emptyset \right\} \cap \left\{\mathcal{H}_i(0) \cup \mathcal{H}_j(0) \neq \emptyset\right\}\right].
		% \end{align}
		% \begin{align}
		% 	\prob[\mathcal{H}_i \cap \mathcal{H}_j \neq \emptyset | X_i = 1] &\leq n C_z\prob\left[\left\{\hat{\mathcal{H}}_i \cap \hat{\mathcal{H}}_j \neq \emptyset \right\} \cap \left\{X_i = 1\right\}\right],\\
		% 	&\leq n C_z\prob\left[\left\{\hat{\mathcal{H}}_i \cap \hat{\mathcal{H}}_j \neq \emptyset \right\} \cap \left(\left\{X_i = 1\right\} \cup \left\{X_i = 1\right\}\right)\right].
		% \end{align}
		% Now since each update in our histories can never remove more than one vertex at a time, if $\hat{\mathcal{H}}_i(0) \cup \hat{\mathcal{H}}_j(0) = \emptyset$, then there must be some time $t > 0$ at which $\hat{\mathcal{H}}_i(t) \cup \hat{\mathcal{H}}_j(t) = \{w\}$ for a single vertex $w$. Proceeding backwards from $t_*$, define $S$ to be the first such time this happens, or put $S = 0$ if the histories always contain at least two vertices.
		Proceeding backwards from $t_*$, define $S$ to be the random time at which $\mathcal{H}_i(t) \cup \mathcal{H}_j(t)$ first reduced to less than two vertices, or define $S = 0$ if the combined histories contain at least two vertices all the way to time $0$. (We cannot say that $S$ is the random time at which $\hist_i(t) \cup \hist_j(t)$ first reduces to a single vertex since a single update may remove more than one vertex. See Figure \ref{fig:nonoblivious shrink} for an example.)
		Note that
		\begin{equation}
			\label{eq:prob union subset of}
			\{\hist_i(0) \cup \hist_j(0) \neq \emptyset\} \subseteq \{\mathscr{F}(v, 0, S) \neq \emptyset\},
		\end{equation}
		where $v$ is either the single vertex $v = \hist_i(S) \cup\hist_j(S)$ in the case that the histories coalesce to a single point at $S$, or any arbitrary single vertex otherwise.
		We also note that
		\begin{equation}
			\label{eq:prob intersect subset of}
			\{\hist_i \cap \hist_j \neq \emptyset\} \subseteq \{\chi\left((\hist_i \cup \hist_j)\cap V \times [S, t_*]\right) \geq k \},
		\end{equation}
		since there must be at least $k$ spatial edges for the histories to meet. The event on the right hand side of \eqref{eq:prob intersect subset of} depends only on the update sequence in $[S, t_*]$. The event on the right hand side of \eqref{eq:prob union subset of} depends only on the update sequence in $[0, S)$. Therefore, given $S$, these events are independent and
		\begin{align}
			&\prob[\left\{\mathcal{H}_i \cap \mathcal{H}_j \neq \emptyset\right\} \cap \left\{\mathcal{H}_i(0) \cup \mathcal{H}_j(0) \neq \emptyset\right\}| S = t_s] \notag \\
			&\phantom{=}\leq \prob[\mathscr{F}(v, 0, S) \neq \emptyset | S = t_s] \prob[\chi\left((\hist_i \cup \hist_j)\cap V \times [S, t_*]\right) \geq k | S = t_s]\\
			&\phantom{=}= m_{t_s}  \prob[\chi\left((\hist_i \cup \hist_j)\cap V \times [S, t_*]\right) \geq k  | S = t_s]
		\end{align}
		and so
		\begin{align}
			\prob[\left\{\mathcal{H}_i \cap \mathcal{H}_j \neq \emptyset\right\} \cap \left\{\mathcal{H}_i(0) \cup \mathcal{H}_j(0) \neq \emptyset\right\}] &\leq \expect[\indicator_{\chi\left((\hist_i \cup \hist_j)\cap V \times [S, t_*]\right) \geq k  }m_S]\\
			&\leq \expect[\indicator_{\chi(\hist_i \cup \hist_j) \geq k } m_S].
		\end{align}
		% If $\mathcal{H}_i(S) \cup \mathcal{H}_j(S)$ contains exactly one vertex then the probability of survival from time $S$ to time $0$ is simply the magnetisation $m_S$. Otherwise, either $S = 0$ or $|\mathcal{H}_i(S) \cup \mathcal{H}_j(S)| = 0$. In the latter case the histories cannot reach time $0$ and in the former, 
		% \begin{equation}
		% 	\prob\left[\mathcal{H}_i(0) \cup \mathcal{H}_j(0) \neq \emptyset | S = 0\right] = 1 = m_{0}.
		% \end{equation}
		% Overall,
		% \begin{equation}
		% 	\prob\left[\mathcal{H}_i(0) \cup \mathcal{H}_j(0) \neq \emptyset | S = t_s\right] \leq m_{t_s}.
		% \end{equation}
		From Corollary \ref{cor:exponential decay magnetization}, 
		\begin{equation}
			m_S \leq \euler^{t_* - S} m_{t_*}
		\end{equation}
		and since $|\hist_i(t) \cup \hist_j(t)| \geq 2$ for $t \in (S, t_*]$,
		\begin{equation}
			t_* - S \leq \mathcal{L}(\hist_i \cup \hist_j)/2.
		\end{equation}
		So
		\begin{align}
			\prob[\mathcal{H}_i \cap \mathcal{H}_j \neq \emptyset | X_i = 1] %&\leq m_{t_*}^{-1} \expect[\indicator_{\hat{\mathcal{H}}_i \cap \hat{\mathcal{H}}_j \neq \emptyset} m_S] \\
			% &\leq m_{t_*}^{-1} \expect[\indicator_{\chi(\mathcal{H}_{ij}) \geq k-1} m_S]\\
			&\leq m_{t_*}^{-1} m_{t_*} \expect[\indicator_{\chi(\hist_i \cup \hist_j) \geq k} \euler^{\mathcal{L}(\hist_i\cup \hist_j)/2}]\\
			% &= \expect[\indicator_{\chi(\mathcal{H}_{ij}) \geq k-1} \euler^{\mathcal{L}(\hist_i\cup \hist_j)/2}]\\
			&\leq \expect[\euler^{\chi(\hist_i\cup \hist_j) - k} \euler^{\mathcal{L}(\hist_i\cup \hist_j)/2}]\\
			% &\leq \euler^{-(k-1)} \expect[\euler^{\chi(\mathcal{H}_{ij}) + (1/2)L(\mathcal{H}_{ij}) + 1} ]\\
			&= \euler^{-k} \expect[\euler^{\chi(\hist_i\cup \hist_j) + \mathcal{L}(\hist_i\cup \hist_j)/2} ].
		\end{align}

		% [OPTIMIZE ALPHA?]
		
		From Lemma \ref{lem:full submartingale thing}, for any $\alpha > \ln(2)$ if
		\begin{equation}
			\tanh(\beta\Delta) \leq \frac{1-2\euler^{-\alpha}}{2(\euler^{\Delta(\alpha+1)} - \euler^{-\alpha})}
		\end{equation}
		then
		\begin{align}
			\expect[\euler^{\chi(\hist_i\cup \hist_j) + \mathcal{L}(\hist_i\cup \hist_j)/2} ] &\leq \euler^{2\alpha}
		\end{align}
		and we get the desired result by choosing $C = \exp(2\alpha)$.
	\end{proof}

	The last of our additional lemmas comes from \cite[Lemma 3.1]{Lubetzky2015-po}. We have made our statement of the lemma slightly more precise and so we have rewritten both the lemma and proof out here along with our modifications. In particular, we have specified precisely how small $\beta$ must be for the statement to hold.

	\begin{lemma}[\cite{Lubetzky2015-po}, Lemma 3.1]
		\label{lem:full submartingale thing}
		Consider the update histories for the Ising heat-bath dynamics on a graph $G = (V, E)$ with fixed degree $\Delta$. For any $0 \leq \eta < 1$, $\lambda \in \mathbb{R}$, and $\alpha > -\ln(1 - \eta)$, if
		\begin{equation}
			\tanh(\beta\Delta) \leq \frac{1 - \eta - \euler^{-\alpha}}{\euler^{(\alpha+\lambda)\Delta} - \euler^{-\alpha}},
		\end{equation}
		then for any $A \subseteq V$,
		\begin{equation}
			\expect[\exp(\lambda \chi(\mathcal{H}_A) + \eta \mathcal{L}(\mathcal{H}_A))] \leq \exp(\alpha |A|).
		\end{equation}
	\end{lemma}
	\begin{proof}
		We first relax our histories to our alternative construction by observing that
		\begin{align}
			&\chi(\mathcal{H}_A) \leq \chi(\hat{\mathcal{H}}_A), &\mathcal{L}(\mathcal{H}_A) \leq \mathcal{L}(\hat{\mathcal{H}}_A).
		\end{align}


		Let $W_s = |\hat{\mathcal{H}}_A (t_* - s)|$, let $Y_s = \chi(\hat{\mathcal{H}}_A \cap V \times [t_* - s, t_*])$ count the total number of spatial edges observed in the history by time $t_* - s$ and let $Z_s = \mathcal{L}(\hat{\mathcal{H}}_A \cap V \times [t_* - s, t_*])$. 

		Initially, $W_0 = |A|$, $Y_0 = 0$, and $Z_0 = 0$. Recall that an oblivious update of a vertex causes it to be removed from the history and that a non-oblivious update causes the history to branch out to its $\Delta$ neighbours. Oblivious updates occur at rate $\theta W_s$ and cause $W_s$ to decrease by $1$. Non-oblivious updates occur at rate $(1 - \theta)W_s$ and cause both $W_s$ and $Y_s$ to increase by no more than $\Delta$. The length, $Z_s$, grows as $\intd Z_s = W_s \intd s$. Therefore we can create a coupled process $(\bar{W}_s, \bar{Y}_s, \bar{Z}_s)$ such that $\bar{W}_s \geq W_s$, $\bar{Y}_s \geq Y_s$, and $\bar{Z}_s \geq Z_s$ in the following way. We start with $(\bar{W}_s, \bar{Y}_s, \bar{Z}_s) = (|A|, 0, 0)$ and at rate $\theta \bar{W}_s$, $\bar{W}_s$ decreases by $1$; at rate $(1 - \theta)\bar{W}_s$, both $\bar{W}_s$ and $\bar{Y}_s$ increase by $\Delta$; and $\bar{Z}_s$ grows as $\intd \bar{Z}_s = \bar{W}_s \intd s$.

		Let $Q_s = \exp \left(\alpha \bar{W}_s + \lambda \bar{Y}_s + \eta \bar{Z}_s\right)$ where $\alpha$, $\lambda$, and $\eta$ are some fixed constants yet to be determined, and $\alpha > -\ln(1 - \eta)$. 
		% We will show that $Q_s$ is a supermartingale. Let $h$ be some small time-step. Then [FIX THIS]
		% \begin{align}
		% 	\expect\left[Q_{s_0 + h} - Q_{s_0}| Q_{s_0}\right] &= h \theta \bar{W}_{s_0} \left(\exp(\alpha (\bar{W}_{s_0} - 1) + \lambda \bar{Y}_{s_0}) - \exp(\alpha \bar{W}_{s_0} + \lambda \bar{Y}_{s_0})\right) + \\
		% 	&\hphantom{= .} h (1 - \theta) \bar{W}_{s_0} \left( \exp(\alpha (\bar{W}_{s_0} + \Delta) + \lambda (\bar{Y}_{s_0} + \Delta)) - \exp(\alpha \bar{W}_{s_0} + \lambda \bar{Y}_{s_0}) \right) + \notag\\
		% 	&\hphantom{= .} \mathcal{O}(h^2) \notag\\
		% 	&= \left(\eta + \theta (\euler^{-\alpha} - 1) + (1 - \theta)(\euler^{(\alpha + \lambda)\Delta} - 1)\right)h \bar{W}_{s_0} Q_{s_0} + \mathcal{O}(h^2).
		% \end{align}
		% Dividing through by $h$ and taking $h$ to $0$, we have
		We have
		\begin{equation}
			\label{eq:martingalederivative}
			\left.\frac{d}{ds} \expect\left[Q_{s} | Q_{s_0}\right] \right|_{s = s_0} = \left(\eta + \theta (\euler^{-\alpha} - 1) + (1 - \theta)(\euler^{(\alpha + \lambda)\Delta} - 1)\right)\bar{W}_{s_0} Q_{s_0},
		\end{equation}
		which is non-positive when
		\begin{equation}
			\theta \geq \frac{\eta + \euler^{(\alpha + \lambda)\Delta} - 1}{\euler^{(\alpha + \lambda)\Delta} - \euler^{-\alpha}}
		\end{equation}
		or in terms of the inverse temperature, $\beta$, 
		\begin{equation}
			\label{eq:boundbeta for supermartingale}
			\tanh(\beta\Delta) \leq \frac{1 - \eta - \euler^{-\alpha}}{\euler^{(\alpha+\lambda)\Delta} - \euler^{-\alpha}}.
		\end{equation}
		Hence $Q_s$ is a supermartingale when \eqref{eq:boundbeta for supermartingale} holds.
		Define the stopping time 
		\begin{equation}
			\tau = \inf\{s: \bar{W}_s = 0\}.
		\end{equation}
		At this time, the histories have completely died out and $\bar{Y}_s$ and $\bar{Z}_s$ cannot grow any more. That is, $\bar{Y}_\tau \geq \chi(\hist_A)$ and $\bar{Z}_\tau \geq \mathcal{L}(\hist_A)$, and so $\expect\exp(\lambda \chi(\hist_A) + \eta \mathcal{L}(\hist_A)) \leq \expect Q_\tau$. From optional stopping,
		\begin{align}
			\expect[Q_\tau] &\leq \expect[Q_{0}]\\
			&= \exp(\alpha |A|).
		\end{align}
	\end{proof}