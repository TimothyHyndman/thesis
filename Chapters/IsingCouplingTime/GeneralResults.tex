%!TEX root = ..\..\main.tex
\chapter{The Coupling Time on Vertex Transitive Graphs}
\label{Ch:GeneralResults}

\lhead{Chapter \ref{Ch:GeneralResults}. \emph{The Coupling Time on Vertex Transitive Graphs}}
	
In this chapter we prove the following theorem.

\begin{conjecture}
\label{thm:Coupling Distribution on Lattice}
	Let $T_L$ be the coupling time for the continuous-time Ising heat-bath 
	dynamics for the zero-field ferromagnetic Ising model on the torus
	$(\mathbb{Z} / L \mathbb{Z})^d$. Then for any small enough 
	inverse-temperature $\beta$,
	\begin{equation}
		\lim_{L \rightarrow \infty} \prob[T_L < a_L z + b_L] = \euler^{-\euler^{-z}}
	\end{equation}
	where $a_L$ and $b_L$ have yet to be determined.
\end{conjecture}
\begin{proof}
\end{proof}

Heuristically, in the high-temperature regime, we expect the dynamics to be similar to those when $\beta = 0$. This is like coupon collector blahblah etc...

\section{Information Percolation in higher dimensions}
In the previous chapter, we showed that on the cycle, there was a coupling that made the update history of a single vertex to be a continuous-time random walk that died at rate $\theta$. On lattices of dimension $d > 2$, we can no longer use this coupling and so the updates histories are significantly more complex. 

Recall from Section \ref{sec: definition update support function} that given a target time $t^*$, the update history of a vertex set $A$ at time $t$, $\mathcal{H}_A(t)$, is the set of vertices whose spins at time $t$ determine the spins of $A$ at time $t^*$. Developing this history backwards in time from $t = t^*$ produces a subgraph of $\Omega \times [0, t^*]$ which we write as $\mathcal{H}_A$ and call the update history of vertex set $A$. This history can be constructed using the update sequence along $(t, t^*]$. 

In practise, we may choose to construct this history as follows: For each $i \in A$, create a temporal edge between $(i, t^*)$ and $(i, t_i)$ where $t_i$ is the time of the latest update to $i$ (or $0$ if $i$ is never updated). Then for each update $(i, u, t_i)$, we either terminate the edge if $u$ is such that the update is oblivious, or we add spatial branches to each of the neighbours of $i$. We repeat this process recursively for the neighbours of $i$ until every branch has been terminated due to an oblivious update or has reached time $0$.

However, it is possible for vertices to be removed from $\mathcal{H}_A(t)$ from updates that are not oblivious. [PUT EXAMPLE IN]. Since our method above for constructing the history does not take this into account, the history it produces will possibly be larger than $\mathcal{H}_A$. To ensure a distinction between the two, the history that results from the above construction we will denote $\hat{\mathcal{H}}_A$, and likewise $\hat{\mathcal{H}}_A(t)$ for the history at time $t$ that results from the above construction. We have that
\begin{equation}
	\mathcal{H}_A(t) \subseteq \hat{\mathcal{H}}_A(t)
\end{equation}
and also that $\mathcal{H}_A$ is a subgraph of $\hat{\mathcal{H}}_A$.

\subsection{Magnetization}
One quantity which we used multiple times in Chapter \ref{Ch:1D} was $\prob[X_i = 1]$. Although it was not required earlier, we would now like to make clear that this is in fact the magnetization at time $t^*$. 

The magnetization at vertex $i \in V$ at time $t > 0$ is defined to be
\begin{equation}
	m_t(i) = \expect[\mathcal{T}_t[i]]
\end{equation}
where $(\mathcal{T}_t)_{t \geq 0}$ is the dynamics starting from the all-plus configuration. %On vertex-transitive graphs, we can drop the vertex specific notation and simply write $m_t$. 
Given a monotonically coupled chain $(\mathcal{B}_t)_{t\geq0}$, starting in the all minus configuration and such that $\mathcal{T}_t[i] \geq \mathcal{B}_t[i]$ for all $t\geq 0$ and $i \in V$, we can split up this expectation by conditioning on the event $A_t = \{\mathcal{T}_t[i] \neq \mathcal{B}_t[i]\}$. We obtain that
\begin{align}
	m_t(i) &= \expect[Y_t^+[i]]\\
	&= \prob\left[A_t\right] \left(\prob\left[Y_t^+[i] = 1 | A_t\right] - \prob\left[Y_t^+[i] = -1|A_t\right]\right)  \\
	&\phantom{= } + \prob\left[A_t^\complement\right] \left( \prob\left[Y_t^+[i] = 1 |A_t^\complement\right] - \prob\left[Y_t^+[i] = -1|A_t^\complement\right] \right). \notag
\end{align}

Now if event $A_t^\complement$ holds, $\mathcal{T}_t[i] = \mathcal{B}_t[i]$, and so by symmetry vertex $i$ must take values $-1$ and $+1$ uniformly. Furthermore, by the monontonicity of our coupling, if $A_t$ holds, we must have that $\mathcal{T}_t[i] = +1$ and $\mathcal{B}_t[i] = -1$.
So
\begin{align}
	m_t(i) = \prob[A_t].
\end{align}
Finally, given a target time $t^*$, $X_i$ is defined such that $\{X_i = 1\} = A_{t^*}$. So 
\begin{equation}
	\label{eq:prob X_i = 1 is m_t*}
	\prob[X_i = 1] = m_{t^*}(i).	
\end{equation}
This motivates the following restatement of part of Lemma 2.1 from \cite{Lubetzky2016-wd}.

\begin{lemma}[\cite{Lubetzky2016-wd}, Lemma 2.1]
	There exist some constant $c_{\beta, d} > 0$ such that for any $t > 0$,
	\begin{equation}
		m_t \leq 2 \euler^{-c_{\beta, d} t}
	\end{equation}
\end{lemma}
\begin{corollary}
	\begin{equation}
		\prob[X_i = 1] \leq 2 \euler^{-c_{\beta, d} t^*}
	\end{equation}
\end{corollary}

\section{Setup}
Define the time
\begin{equation}
	t_c(n) = \inf\left\{ t > 0 : m_t = \frac{1}{n}\right\}.
\end{equation}
Fix $z$ and a time of interest $t^* = t_c(n) + z$.

\begin{lemma}[\cite{Lubetzky2017-nc}, Claim 3.3]
	On any graph with maximum degree $\Delta$, for any $t, s > 0$ we have
	\begin{equation}
		\euler^{-2s} \leq \frac{\sum_i m_{t+s}[i]^2}{\sum_i m_t[i]^2} \leq \euler^{-2(1 - \beta \Delta)s}.
	\end{equation}
\end{lemma}
The following corollaries are then straightfoward.
\begin{corollary}
	\label{cor:exponential decay magnetization}
	On any vertex transitive graph with degree $\Delta$, for any $t, s > 0$ we have
	\begin{equation}
		\euler^{-s} m_t \leq m_{t+s} \leq m_t \euler^{-(1 - \beta \Delta)s}.
	\end{equation}
\end{corollary}
\begin{corollary}
\label{cor:magnetization of t star}
	On any vertex transitive graph with degree $\Delta$, $m_{t^*}$ can be bounded as follows:

	For $z \geq 0$,
	\begin{equation}
		\frac{\euler^{-z}}{n} \leq m_{t^*} \leq \frac{\euler^{-(1 - \beta \Delta)z}}{n}.
	\end{equation}

	For $z \leq 0$,
	\begin{equation}
		\frac{\euler^{-(1 - \beta \Delta)z}}{n} \leq m_{t^*} \leq \frac{\euler^{-z}}{n}.
	\end{equation}
\end{corollary}

\begin{corollary}
	On any vertex transitive graph with degree $\Delta$, for $\beta < 1/\Delta$
	\begin{equation}
		\ln(n) \leq t_c(n) \leq \frac{\ln(n)}{1 - \beta \Delta}
	\end{equation}
\end{corollary}

\section{Proof of Theorem \ref{thm:Coupling Distribution on Lattice}}
\begin{lemma}
\label{lem:nDlambda}
	Using the above setup
	\begin{equation}
		 \leq \lim_{n \rightarrow \infty} \lambda \leq n m_{t^*}
	\end{equation}
\end{lemma}
\begin{proof}
	\begin{align}
		\lambda &= \sum_{i \in V} \expect\left[\frac{X_i}{X_i + U_i} \indicator[ X_i + U_i \geq 1] \right]\\
			&= \sum_{i = 1}^n \prob(X_i = 1) \expect \left[\frac{1}{1 + U_i}| X_i = 1\right]\\
			&= n m_{t^*} \expect \left[\frac{1}{1 + U_i}| X_i = 1\right]
	\end{align}
	where we have used that $X_i$ is zero-one, \eqref{eq:prob X_i = 1 is m_t*}, and the transitivity of the graph.
	Clearly 
	\begin{align}
		\expect \left[\frac{1}{1 + U_i}| X_i = 1\right] \leq 1
	\end{align}
	and so $\lambda \leq n m_{t^*}$.

	By Jensen's inequality
	\begin{align}
		\expect \left[\frac{1}{1 + U_i}| X_i = 1\right] &\geq \frac{1}{\expect[1 + U_i | X_i = 1]}\\
			&= \frac{1}{1 + \expect[U_i | X_i = 1]}.
	\end{align}
	so in order to find a lower bound for $\lambda$ we will find an upper bound to $\expect[U_i | X_i = 1]$. Now
	\begin{align}
		\expect[U_i | X_i = 1] &= \sum_{j \in B_i} \prob[X_j = 1| X_i = 1]\\
			&= \sum_{k=1}^{\lfloor b_n \rfloor} \sum_{|j - i| = k} \prob[X_j = 1| X_i = 1]\\
			& \leq \sum_{k=1}^{\lfloor b_n \rfloor} \sum_{|j - i| = k} \left(C_{z, \epsilon} \euler^{2 \alpha} n^{-\epsilon} + \euler^{-k + 2 + 2\alpha} \right)
	\end{align}
	by [BLAH]. From [BLAH]
	\begin{align}
		\expect[U_i | X_i = 1] &\leq \sum_{k=1}^{\lfloor b_n \rfloor} P(k) \left(C_{z, \epsilon} \euler^{2 \alpha} n^{-\epsilon} + \euler^{-k + 2 + 2\alpha} \right)\\
		&\leq C_{z, \epsilon} \euler^{2 \alpha} b_n P(b_n) n^{-\epsilon} + \euler^{2 + 2\alpha}\sum_{k=1}^\infty P(k)\euler^{-k}.
	\end{align}
	As $n \rightarrow \infty$, the first term vanishes and we are left with [COMPLETE]
\end{proof}

\begin{lemma}
\label{lem:delta1 goes to 0 general}
	Let $\delta_1$ be as defined above in [REF]. Then
	\begin{equation}
		\lim_{n\rightarrow\infty} \delta_1 = 0.
	\end{equation}
\end{lemma}
\begin{proof}
	Starting with the definition of $\delta_1$, we have
	\begin{align}
		\delta_1 &= \sum_{i = 1}^n \sum_{k = 0}^{|B_i|} \prob[X_i = 1, U_i = k] \expect \left| \frac{\prob[X_i = 1, U_i = k|W_i]}{\prob[X_i = 1, U_i = k]} - 1 \right|\\
		&= n \sum_{k = 0}^{|B_i|} \expect \left|\prob[X_i = 1, U_i = k|W_i] - \prob[X_i = 1, U_i = k] \right|%\\
		\label{eq:nD delta1 line 2}
		% &\leq n \sup_{W_i} \sum_{k = 0}^{2 b_n} \left| \prob[X_i = 1, U_i = k|W_i] - \prob[X_i = 1, U_i = k] \right|
	\end{align}
	by the transitivity of the graph.
	Let
	\begin{equation}
		C_i^c = \{j : |j - i| \leq (c_n + b_n)/2\}
	\end{equation}
	be the set of vertices within distance $(b_n + c_n)/2$ of $i$ and define the events
	\begin{equation}
		A_1 = \{\exists j \in B_i \cup \{i\}, \exists t \in [0, t^*] : \mathcal{H}_j(t) \nsubseteq  C_i^c\}
	\end{equation}
	and
	\begin{equation}
		A_2 = \{\exists j \in D_i, \exists t \in [0, t^*] : \mathcal{H}_j(t) \cap C_i^c \neq \emptyset\}
	\end{equation}
	as well as their intersection
	\begin{equation}
		A = A_1 \cap A_2.
	\end{equation}

	From Lemma \ref{lem:conditioning on A removes conditioning on W},
	\begin{equation}
		\prob[X_i = 1, U_i = j | A^\complement, W_i] = \prob[X_i = 1, U_i = j | A^\complement].
	\end{equation} 
	Continuing on from \eqref{eq:nD delta1 line 2}, we split the probabilities into
	\begin{align}
		\delta_1 &= n \sum_{k = 0}^{|B_i|} \expect \left| \prob[X_i = 1, U_i = k|W_i, A]\prob[A|W_i] - \prob[X_i = 1, U_i = k| A]\prob[A] + \vphantom{A^\complement} \right.\\
		&\hphantom{= .} \left.\prob(X_i = 1, U_i = k | A^\complement) (\prob[A^\complement|W_i] - \prob[A^\complement]) \right| \notag \\
		&\leq n (|B_i| + 1) \expect\left[ \prob[A|W_i] + \prob[A] + \left|\prob[A^\complement|W_i] - \prob[A^\complement]\right|\right]\\
		&= n (|B_i| + 1) \expect\left[ \prob[A|W_i] + \prob[A] + \left|1 - \prob[A|W_i] - (1 - \prob[A])\right|\right]\\
		&\leq n (|B_i| + 1) \expect\left[ \prob[A|W_i] + \prob[A] + \prob[A|W_i] + \prob[A])\right]\\
		&= 2 n (|B_i| + 1) \left(\expect[\prob[A|W_i]] + \prob[A]\right)\\
		&= 4 n (|B_i| + 1) \prob[A]
	\end{align}

	For either $A_1$ or $A_2$ to hold, there must exists a history that spreads at least distance $(c_n - b_n) / 2$ away from its starting vertex. 
	By a union bound
	\begin{align}
		\prob[A] &\leq \sum_{j = 1}^n \prob[\mathcal{H}_i \nsubseteq B(i, (c_n - b_n)/2) \times [0, t^*]]\\
			&= n \prob\left[\bigcup_{u \in [0, t^*]}\mathcal{H}_i(t^* - u) \nsubseteq B(i, (c_n - b_n)/2)\right]
	\end{align}
	Combining this with Lemma \ref{lem:prob history contained in ball}, and recalling our choices of $b_n = \ln(L)$ and $c_n = \ln(L)^2$ we get that
	\begin{align}
		\delta_1 &\leq 4 n^2 (|B_i| + 1) \exp(t^* \Delta - \ln\Delta(c_n - b_n)/2)\\
		&\leq 4n^{2 + \Delta/(1 - \beta \Delta)}(|B_i| + 1)\exp(\Delta z) \exp(-\ln \Delta (c_n - b_n)/2)
	\end{align}
	which goes to $0$ as $n \rightarrow \infty$.

	% By a union bound,
	% \begin{align}
	% 	\delta_1 &\leq 4 n (|B_i| + 1) \sum_{j \in B_i \cup \{i\}} \sum_{k \in D_i} \prob[\mathcal{H}_{j} \cap \mathcal{H}_k \neq \emptyset].
	% \end{align}

	% [FIX THIS LEMMA]
	% % From Lemma \ref{lem:intersecting histories bound},
	% \begin{align}
	% 	\delta_1 &\leq 4 n^2 (|B_i| + 1)^2 \exp(2\alpha) \exp(-\lambda(c_n - b_n)).
	% \end{align}

	% On the torus $(\mathbb{Z} / L \mathbb{Z})^d$, $n = L^d$, $c_n = \log(L)^2$, $b_n = \log(L)$, and $|B_i| \leq C \log(L)^d$. So

	% \begin{equation}
	% 	\delta_1 \leq C L^{2d} \log(L)^{2d} \exp(-\lambda(\log(L)^2 - \log(L))
	% \end{equation}
	% which goes to $0$ as $L \rightarrow \infty$.
\end{proof}

\begin{lemma}
\label{lem:delta4 goes to 0 general}
	Let $\delta_4$ be as defined above in [REF]. Then
	\begin{equation}
		\lim_{n\rightarrow\infty} \delta_4 = 0.
	\end{equation}
\end{lemma}
\begin{proof}
	Starting with the definition of $\delta_4$, we have
	\begin{align}
		\delta_4 &= \sum_{i = 1}^n \left(\expect[X_i Z_i] + \expect[X_i]\expect[X_i + U_i + Z_i]\right)\\
			&= n \expect[X_i Z_i] + n m_{t^*}^2 \left(1 + |B_i| + |C_i|\right)\\
			&= n m_{t^*} \expect[Z_i | X_i = 1] + n m_{t^*}^2 \left(1 + |B_i| + |C_i|\right)\\
			&\leq C_z \expect[Z_i | X_i = 1] + \frac{C_z^2}{n} \left(1 + |B_i| + |C_i|\right)
			% &= n m_{t^*} \left(\sum_{j \in C_i} \prob[X_j = 1 | X_i = 1] + m_{t^*} \left(1 + |B_i| + |C_i|\right) \right)%\\
			% &\leq n m_{t^*} \left(\sum_{j \in C_i} \left(m_{t^*} + \prob[\mathcal{H}_{j} \cap \mathcal{H}_i \neq \emptyset | X_i = 1\right) + m_{t^*} \left(1 + |B_i| + |C_i|\right) \right)
	\end{align}
	where 
	\begin{equation}
		C_z = \max(\euler^{-z}, \euler^{-(1 - \beta \Delta)z}).
	\end{equation}
	Now from Lemmas \ref{lem:size B_i} and \ref{lem:size C_i}, the second term above vanishes as $n \rightarrow \infty$. So we turn our attention to the first term. From Lemma \ref{lem:nd X_j given X_i}, there exists a $C > 0$ such that for small enough $\beta$,
	\begin{align}
		\expect[Z_i | X_i = 1] &= \sum_{j \in C_i} \prob[X_j = 1 | X_i = 1]\\
			% &= \sum_{k = \lfloor b_n \rfloor + 1}^{\lfloor c_n \rfloor} \sum_{j: |i - j| = k} \prob[X_j = 1 | X_i = 1]\\
			&\leq |C_i| \left(m_{t^*} + C \euler^{-b_n}\right)\\
			&\leq |C_i| \left(\frac{C_z}{n} + \frac{C}{n}\right)
	\end{align}
	which goes to zero as $n \rightarrow \infty$.
	% NEED TO PROVE LEMMA \ref{lem:intersecting histories bound given X_j = 1}
\end{proof}

\section{Additional Lemmas}
We introduce some additional notation in this section
[DEFINE $\chi$, $\mathcal{L}$]

The first of our additional Lemmas comes from \cite[Lemma 3.1]{Lubetzky2014-po}. We have slightly modified the statement and proof.
\begin{lemma}[\cite{Lubetzky2014-po}]
	\label{lem:full submartingale thing}
	Let 
	\begin{equation}
		\chi(\mathcal{H}_i) = \#\left\{ \left( (u,t), (v,t) \right) \in \mathcal{H}_i \right\}
	\end{equation}
	count the total number of horizontal edges in $\mathcal{H}_i$ and let 
	\begin{equation}
		\mathcal{L}(\mathcal{H}_i) = \sum_{i \in V} \int_{0}^{t^*} \indicator_{(i, t) \in \mathcal{H}_i} \intd t
	\end{equation}
	be the sum of the lengths of all the horizontal edges in $\hist_i$. Then for any $0 \leq \eta < 1$, $\lambda \in \mathbb{R}$, $\alpha > -\ln(1 - \eta)$, there exists $\beta_0 = \beta_0(d, \eta, \lambda, \alpha)$ such that if $\beta < \beta_0$ then for any $A \subseteq V$,
	\begin{equation}
		\expect[\lambda \exp(\chi(\mathcal{H}_A) + \eta \mathcal{L}(\mathcal{H}_A))] \leq \exp(\alpha |A|)
	\end{equation}
	where $\hist_A$ is defined as
	\begin{equation}
		\hist_A = \bigcup_{i \in A} \hist_i.
	\end{equation}
\end{lemma}
\begin{proof}
	We first relax our histories to our alternative construction by observing that
	\begin{align}
		&\chi(\mathcal{H}_i) \leq \chi(\hat{\mathcal{H}}_i), &Z(\mathcal{H}_i) \leq Z(\hat{\mathcal{H}}_i).
	\end{align}


	Let $W_s = |\hat{\mathcal{H}}_i (t^* - s)|$, let $Y_s = \chi(\hat{\mathcal{H}}_i \cap V \times [t^* - s, t^*])$ count the total number of spatial edges observed in the history by time $t^* - s$ and let $Z_s = \mathcal{L}(\hat{\mathcal{H}}_i \cap V \times [t^* - s, t^*])$. 

	Initially, $W_0 = 1$, $Y_0 = 0$, and $Z_0 = 0$. Recall that an oblivious update of a vertex causes it to be removed from the history and that a non-oblivious update causes the history to branch out to its $\Delta$ neighbours. Oblivious updates occur at rate $\theta W_s$ and cause $W_s$ to decrease by $1$. Non-oblivious updates occur at rate $(1 - \theta)W_s$ and cause both $W_s$ and $Y_s$ to increase by no more than $\Delta$. The length, $Z_s$, grows as $\intd Z_s = W_s \intd s$. Therefore we can create a coupled process $(\bar{W}_s, \bar{Y}_s, \bar{Z}_s)$ such that $\bar{W}_s \geq W_s$, $\bar{Y}_s \geq Y_s$, and $\bar{Z}_s \geq Z_s$ in the following way. We start with $(\bar{W}_s, \bar{Y}_s, \bar{Z}_s) = (|A|, 0, 0)$ and at rate $\theta \bar{W}_s$, $\bar{W}_s$ decreases by $1$; at rate $(1 - \theta)\bar{W}_s$, both $\bar{W}_s$ and $\bar{Y}_s$ increase by $\Delta$; and $\bar{Z}_s$ grows as $\intd \bar{Z}_s = \bar{W}_s \intd s$.

	Let $Q_s = \exp \left(\alpha \bar{W}_s + \lambda \bar{Y}_s + \eta \bar{Z}_s\right)$ where $\alpha$, $\lambda$, and $\eta$ are some fixed constants yet to be determined, and $\alpha > -\ln(1 - \eta)$. We will show that $Q_s$ is a supermartingale. Let $h$ be some small time-step. Then [FIX THIS]
	\begin{align}
		\expect\left[Q_{s_0 + h} - Q_{s_0}| Q_{s_0}\right] &= h \theta \bar{W}_{s_0} \left(\exp(\alpha (\bar{W}_{s_0} - 1) + \lambda \bar{Y}_{s_0}) - \exp(\alpha \bar{W}_{s_0} + \lambda \bar{Y}_{s_0})\right) + \\
		&\hphantom{= .} h (1 - \theta) \bar{W}_{s_0} \left( \exp(\alpha (\bar{W}_{s_0} + \Delta) + \lambda (\bar{Y}_{s_0} + \Delta)) - \exp(\alpha \bar{W}_{s_0} + \lambda \bar{Y}_{s_0}) \right) + \notag\\
		&\hphantom{= .} \mathcal{O}(h^2) \notag\\
		&= \left(\eta + \theta (\euler^{-\alpha} - 1) + (1 - \theta)(\euler^{(\alpha + \lambda)\Delta} - 1)\right)h \bar{W}_{s_0} Q_{s_0} + \mathcal{O}(h^2).
	\end{align}
	Dividing through by $h$ and taking $h$ to $0$, we have
	\begin{equation}
		\left.\frac{d}{ds} \expect\left[Q_{s} | Q_{s_0}\right] \right|_{s = s_0} = \left(\eta + \theta (\euler^{-\alpha} - 1) + (1 - \theta)(\euler^{(\alpha + \lambda)\Delta} - 1)\right)\bar{W}_{s_0} Q_{s_0}
	\end{equation}
	which is negative for $\theta$ sufficiently close to $1$. Hence $Q_s$ is a supermartingale for sufficiently large $\theta$.

	Define the stopping time 
	\begin{equation}
		\tau = \inf\{s: \bar{W}_s = 0\}.
	\end{equation}
	Finally, from optional stopping [FIGURE THIS OUT PROPERLY],
	\begin{align}
		\expect[\exp(\lambda \bar{Y}_\tau  + \eta \bar{Z}_\tau)] &\leq \expect[Q_{0}]\\
		&= \exp(\alpha |A|).
	\end{align}
\end{proof}
% \begin{lemma}
% 	\label{lem:intersecting histories bound}
% 	\begin{equation}
% 		\prob[\mathcal{H}_{j} \cap \mathcal{H}_k \neq \emptyset] \leq \exp(2\alpha) \exp(-\lambda |j - k|).
% 	\end{equation}
% \end{lemma}
% \begin{proof}
% 	The following is based on the proof used in Section 3.2 of \cite{Lubetzky2014-po}.

% 	We first relax our histories to our alternative construction by observing that
% 	\begin{equation}
% 		\prob[\mathcal{H}_j \cap \mathcal{H}_k \neq \emptyset] \leq \prob[\hat{\mathcal{H}}_j \cap \hat{\mathcal{H}}_k \neq \emptyset].
% 	\end{equation}

% 	Let $W_s = |\hat{\mathcal{H}}_{\{j, k\}} (t^* - s)|$ and let $Y_s = \#\left\{ \left( (u,t), (v,t) \right) \in \hat{\mathcal{H}}_{\{j, k\}}: t \in [t^* - s, t^*) \right\}$ count the total number of spatial edges observed in the history by time $t^* - s$. For $\hat{\mathcal{H}}_j$ and $\hat{\mathcal{H}}_k$ to intersect at time $t^* - s$, there must be enough spatial edges for the histories to reach each other. That is, we require that
% 	\begin{equation}
% 		Y_s \geq |i - j|.
% 	\end{equation}

% 	Initially, $W_0 = 2$ and $Y_0 = 0$. Recall that an oblivious update of a vertex causes it to be removed from the history and that a non-oblivious update causes the history to branch out to its $\Delta$ neighbours. Oblivious updates occur at rate $\theta W_s$ and cause $W_s$ to decrease by $1$. Non-oblivious updates occur at rate $(1 - \theta)W_s$ and cause both $W_s$ and $Y_s$ to increase by no more than $\Delta$. Therefore we can create a coupled process $(\bar{W}_s, \bar{Y}_s)$ such that $\bar{W}_s \geq W_s$ and $\bar{Y}_s \geq Y_s$ in the following way. We start with $(\bar{W}_s, \bar{Y}_s) = (2, 0)$ and at rate $\theta \bar{W}_s$, $\bar{W}_s$ decreases by $1$, and at rate $(1 - \theta)\bar{W}_s$, both $\bar{W}_s$ and $\bar{Y}_s$ increase by $\Delta$.

% 	Let $Q_s = \exp \left(\alpha \bar{W}_s + \lambda \bar{Y}_s \right)$ where $\alpha$ and $\lambda$ are some fixed constants yet to be determined. We will show that $Q_s$ is a supermartingale. Let $h$ be some small time-step. Then
% 	\begin{align}
% 		\expect\left[Q_{s_0 + h} - Q_{s_0}| Q_{s_0}\right] &= h \theta \bar{W}_{s_0} \left(\exp(\alpha (\bar{W}_{s_0} - 1) + \lambda \bar{Y}_{s_0}) - \exp(\alpha \bar{W}_{s_0} + \lambda \bar{Y}_{s_0})\right) + \\
% 		&\hphantom{= .} h (1 - \theta) \bar{W}_{s_0} \left( \exp(\alpha (\bar{W}_{s_0} + \Delta) + \lambda (\bar{Y}_{s_0} + \Delta)) - \exp(\alpha \bar{W}_{s_0} + \lambda \bar{Y}_{s_0}) \right) + \notag\\
% 		&\hphantom{= .} \mathcal{O}(h^2) \notag\\
% 		&= \left(\theta (\euler^{-\alpha} - 1) + (1 - \theta)(\euler^{(\alpha + \lambda)\Delta} - 1)\right)h \bar{W}_{s_0} Q_{s_0} + \mathcal{O}(h^2).
% 	\end{align}
% 	Dividing through by $h$ and taking $h$ to $0$, we have
% 	\begin{equation}
% 		\left.\frac{d}{ds} \expect\left[Q_{s} | Q_{s_0}\right] \right|_{s = s_0} = \left(\theta (\euler^{-\alpha} - 1) + (1 - \theta)(\euler^{(\alpha + \lambda)\Delta} - 1)\right)\bar{W}_{s_0} Q_{s_0}
% 	\end{equation}
% 	which is negative for $\theta$ sufficiently close to $1$. Hence $Q_s$ is a supermartingale for sufficiently large $\theta$.

% 	Define the stopping time 
% 	\begin{equation}
% 		\tau = \inf\{s: \bar{W}_s = 0\}.
% 	\end{equation}
% 	Since $(\bar{W}_s, \bar{Y}_s) \geq (W_s, Y_s)$, we have that $W_\tau = 0$ which corresponds to the event that both histories, $\mathcal{H}_j$ and $\mathcal{H}_k$, have terminated by time $\tau$. Hence
% 	\begin{equation}
% 		\prob[\hat{\mathcal{H}}_j \cap \hat{\mathcal{H}}_k \neq \emptyset] \leq \prob[Y_\tau \geq |j-k|].
% 	\end{equation}
% 	Now
% 	\begin{equation}
% 		\expect[\exp(\lambda Y_\tau)] \leq \expect[\exp(\lambda\bar{Y}_\tau)].
% 	\end{equation}
% 	and from optional stopping [FIGURE THIS OUT PROPERLY],
% 	\begin{align}
% 		\expect[\exp(\lambda \bar{Y}_\tau )] &\leq \expect[Q_{0}]\\
% 		&= \exp(2\alpha).
% 	\end{align}
% 	Hence 
% 	\begin{equation}
% 		\prob[Y_\tau \geq |j-k|] \leq \exp(2\alpha) \exp(-\lambda |j - k|)
% 	\end{equation}
% 	and overall we have that
% 	\begin{equation}
% 		\prob[\mathcal{H}_j \cap \mathcal{H}_k \neq \emptyset] \leq \exp(2\alpha) \exp(-\lambda |j - k|).
% 	\end{equation}
% \end{proof}

% \begin{lemma}
% 	\label{lem:history escaping ball bound}
% 	\begin{equation}
% 		\prob[\mathcal{H}_{i} \not\subset B(i, l)] \leq \exp(\alpha) \exp(-\lambda l).
% 	\end{equation}
% \end{lemma}
% \begin{proof}
% The following is based on the proof used in Section 3.2 of \cite{Lubetzky2014-po}.

% We first relax our histories to our alternative construction by observing that
% \begin{equation}
% 	\prob[\mathcal{H}_{i} \not\subset B(i, l)] \leq \prob[\hat{\mathcal{H}}_{i} \not\subset B(i, l)].
% \end{equation}

% Let $W_s = |\hat{\mathcal{H}}_i (t^* - s)|$ and let $Y_s = \#\left\{ \left( (u,t), (v,t) \right) \in \hat{\mathcal{H}}_i: t \in [t^* - s, t^*) \right\}$ count the total number of spatial edges observed in the history by time $t^* - s$. For $\hat{\mathcal{H}}_i$ to extend beyond $B(i, l)$, there must be enough spatial edges. That is, we require that
% \begin{equation}
% 	Y_s \geq l.
% \end{equation}

% Initially, $W_0 = 1$ and $Y_0 = 0$. Recall that an oblivious update of a vertex causes it to be removed from the history and that a non-oblivious update causes the history to branch out to its $\Delta$ neighbours. Oblivious updates occur at rate $\theta W_s$ and cause $W_s$ to decrease by $1$. Non-oblivious updates occur at rate $(1 - \theta)W_s$ and cause both $W_s$ and $Y_s$ to increase by no more than $\Delta$. Therefore we can create a coupled process $(\bar{W}_s, \bar{Y}_s)$ such that $\bar{W}_s \geq W_s$ and $\bar{Y}_s \geq Y_s$ in the following way. We start with $(\bar{W}_s, \bar{Y}_s) = (1, 0)$ and at rate $\theta \bar{W}_s$, $\bar{W}_s$ decreases by $1$, and at rate $(1 - \theta)\bar{W}_s$, both $\bar{W}_s$ and $\bar{Y}_s$ increase by $\Delta$.

% Let $Q_s = \exp \left(\alpha \bar{W}_s + \lambda \bar{Y}_s \right)$ where $\alpha$ and $\lambda$ are some fixed constants yet to be determined. We will show that $Q_s$ is a supermartingale. Let $h$ be some small time-step. Then
% \begin{align}
% 	\expect\left[Q_{s_0 + h} - Q_{s_0}| Q_{s_0}\right] &= h \theta \bar{W}_{s_0} \left(\exp(\alpha (\bar{W}_{s_0} - 1) + \lambda \bar{Y}_{s_0}) - \exp(\alpha \bar{W}_{s_0} + \lambda \bar{Y}_{s_0})\right) + \\
% 	&\hphantom{= .} h (1 - \theta) \bar{W}_{s_0} \left( \exp(\alpha (\bar{W}_{s_0} + \Delta) + \lambda (\bar{Y}_{s_0} + \Delta)) - \exp(\alpha \bar{W}_{s_0} + \lambda \bar{Y}_{s_0}) \right) + \notag\\
% 	&\hphantom{= .} \mathcal{O}(h^2) \notag\\
% 	&= \left(\theta (\euler^{-\alpha} - 1) + (1 - \theta)(\euler^{(\alpha + \lambda)\Delta} - 1)\right)h \bar{W}_{s_0} Q_{s_0} + \mathcal{O}(h^2).
% \end{align}
% Dividing through by $h$ and taking $h$ to $0$, we have
% \begin{equation}
% 	\left.\frac{d}{ds} \expect\left[Q_{s} | Q_{s_0}\right] \right|_{s = s_0} = \left(\theta (\euler^{-\alpha} - 1) + (1 - \theta)(\euler^{(\alpha + \lambda)\Delta} - 1)\right)\bar{W}_{s_0} Q_{s_0}
% \end{equation}
% which is negative for $\theta$ sufficiently close to $1$. Hence $Q_s$ is a supermartingale for sufficiently large $\theta$.

% Define the stopping time 
% \begin{equation}
% 	\tau = \inf\{s: \bar{W}_s = 0\}.
% \end{equation}
% Since $(\bar{W}_s, \bar{Y}_s) \geq (W_s, Y_s)$, we have that $W_\tau = 0$ which corresponds to the event the history, $\hat{\mathcal{H}}_i$, has terminated by time $\tau$. Hence
% \begin{equation}
% 	\prob[\hat{\mathcal{H}}_{i} \not\subset B(i, l)] \leq \prob[Y_\tau \geq l].
% \end{equation}
% Now
% \begin{equation}
% 	\expect[\exp(\lambda Y_\tau)] \leq \expect[\exp(\lambda\bar{Y}_\tau)].
% \end{equation}
% and from optional stopping [FIGURE THIS OUT PROPERLY],
% \begin{align}
% 	\expect[\exp(\lambda \bar{Y}_\tau )] &\leq \expect[Q_{0}]\\
% 	&= \exp(\alpha).
% \end{align}
% Hence 
% \begin{equation}
% 	\prob[Y_\tau \geq l] \leq \exp(\alpha) \exp(-\lambda l)
% \end{equation}
% and overall we have that
% \begin{equation}
% 	\prob[\mathcal{H}_i \not\subset B(i, l)] \leq \exp(\alpha) \exp(-\lambda l).
% \end{equation}
% \end{proof}
% \begin{lemma}
% 	\label{lem:intersecting histories bound given X_j = 1}
% 	\begin{equation}
% 		\prob[\mathcal{H}_{i} \cap \mathcal{H}_j \neq \emptyset | X_i = 1] \leq  2 \exp(z \Delta^2) \max(\euler^{-z}, \euler^{-(1 - \beta\Delta)z})\Delta^{-k/2} n^{\Delta^2 + 1}
% 	\end{equation}
% 	where $k = |i - j|$.
% \end{lemma}
% \begin{proof}
% 	\begin{align}
% 		\prob[\mathcal{H}_{i} \cap \mathcal{H}_j \neq \emptyset | X_i = 1]  &\leq \frac{\prob[\mathcal{H}_{i} \cap \mathcal{H}_j \neq \emptyset] }{\prob[X_i = 1]}\\
% 		&\leq n \max(\euler^{-z}, \euler^{-(1 - \beta\Delta)z}) \prob[\mathcal{H}_{i} \cap \mathcal{H}_j \neq \emptyset].
% 	\end{align}
% 	Now letting $k = |i - j|$ denote the distance between $i$ and $j$,
% 	\begin{equation}
% 		\prob[\mathcal{H}_{i} \cap \mathcal{H}_j \neq \emptyset] \leq 2 \prob[\mathcal{H}_i \nsubseteq B(i, k/2) ]
% 	\end{equation}
% 	since in order for $\mathcal{H}_i$ and $\mathcal{H}_j$ to intersect, at least one of them must extend beyond halfway the distance between $i$ and $j$. From Lemma \ref{cor:prob update function time 0 in B(i, l)},
% 	\begin{align}
% 		\prob[\mathcal{H}_{i} \cap \mathcal{H}_j \neq \emptyset | X_i = 1] &\leq 2 \exp(z \Delta^2) \max(\euler^{-z}, \euler^{-(1 - \beta\Delta)z})\Delta^{-k/2} n^{\Delta^2 + 1}
% 	\end{align}
% \end{proof}

\begin{lemma}
	The number of vertices at distance $k$ on a degree $\Delta$ transitive graph is
	\begin{equation}
		P(k) = ???
	\end{equation}
\end{lemma}
\begin{lemma}
	\label{lem:size B_i}
	The size of $B_i$ is bounded by 
	\begin{equation}
		|B_i| \leq b_n P(b_n)
	\end{equation}
\end{lemma}
\begin{lemma}
	\label{lem:size C_i}
	The size of $C_i$ is bounded by 
	\begin{equation}
		|C_i| \leq c_nP(c_n)
	\end{equation}
\end{lemma}
% \begin{lemma}
% 	On the $d$ dimensional lattice,
% 	\begin{equation}
% 		\sum_{k = b_n}^\infty \prob[\mathcal{H}_{i} \cap \mathcal{H}_j \neq \emptyset | X_i = 1] \leq 
% 	\end{equation}
% \end{lemma}
% \begin{proof}
% 	On the $d$ dimensional torus, the number of vertices at distance $k$ is no more than something like $k^d$ [NEED FIGURE THIS OUT AND PROVE IT].
% 	\begin{align}
% 		\sum_{j \in C_i} \prob[\mathcal{H}_{i} \cap \mathcal{H}_j \neq \emptyset | X_i = 1] &\leq \sum_{k = b_n}^\infty \sum_{j, |j - i| = k} \prob[\mathcal{H}_{i} \cap \mathcal{H}_j \neq \emptyset | X_i = 1] \\
% 		&\leq \sum_{k = b_n}^\infty k^d \prob[\mathcal{H}_{i} \cap \mathcal{H}_j \neq \emptyset | X_i = 1]
% 	\end{align}
% 	From Lemma \ref{lem:intersecting histories bound given X_j = 1},
% 	\begin{align}
% 		\sum_{j \in C_i} \prob[\mathcal{H}_{i} \cap \mathcal{H}_j \neq \emptyset | X_i = 1] &\leq C(z, \Delta) n^{\Delta^2 + 1} \sum_{k = b_n}^\infty k^d \Delta^{-k/2} \\
% 		&\leq C(z, \Delta) n^{\Delta^2 + 1} \sum_{k = b_n}^\infty k^d \exp(-\ln(\Delta) k/2)\\
% 		&\approxeq C(z, \Delta) n^{\Delta^2 + 1} b_n^d \exp(-\ln(\Delta)b_n/2)
% 	\end{align}
% 	which goes to 0 for large enough $b_n$ [FIGURE OUT HOW BIG].
% \end{proof}

% \begin{lemma}
% 	\label{lem:intersecting histories bound given X_j = 1}
% 	\begin{equation}
% 		\prob[\mathcal{H}_{j} \cap \mathcal{H}_k \neq \emptyset | X_j = 1] \leq 
% 	\end{equation}
% \end{lemma}
% \begin{proof}
% The proof here is similar to that of Lemma \ref{lem:intersecting histories bound}. We first relax to our alternative histories
% \begin{equation}
% 	\prob[\mathcal{H}_{j} \cap \mathcal{H}_k \neq \emptyset | X_j = 1] \leq \prob[\hat{\mathcal{H}}_{j} \cap \hat{\mathcal{H}}_k \neq \emptyset | X_j = 1].
% \end{equation}
% By symmetry, 
% \begin{equation}
% 	\prob[\hat{\mathcal{H}}_{j} \cap \hat{\mathcal{H}}_k \neq \emptyset | X_j = 1] = \prob[\hat{\mathcal{H}}_{j} \cap \hat{\mathcal{H}}_k \neq \emptyset | X_k = 1]
% \end{equation}
% and also
% \begin{equation}
% 	\prob[X_j = 1] = \prob[X_k = 1].
% \end{equation}
% So 
% \begin{equation}
% 	\prob[\hat{\mathcal{H}}_{j} \cap \hat{\mathcal{H}}_k \neq \emptyset | X_j = 1] \leq 2 \prob[\hat{\mathcal{H}}_{j} \cap \hat{\mathcal{H}}_k \neq \emptyset | \{X_j = 1\} \cup \{X_k = 1\}].
% \end{equation}

% We now define $W_s$ and $Y_s$ as before, except now conditioned on the event $\{X_j = 1\} \cup \{X_k = 1\}$. The effect of this conditioning is to forbid updates that reduce $W_s$ to $0$. 

% HANDWAVING: If we stop when $W_s = 1$ then we can ignore the conditioning???

% Define the stopping time 
% \begin{equation}
% 	\tau = \inf\{s: \bar{W}_s = 1\}.
% \end{equation}

% \end{proof}

In \cite{Lubetzky2016-wd}, \citeauthor{Lubetzky2016-wd} proved something similar...
	\begin{lemma}
	\label{lem:prob history contained in ball}
		Let $B(i, l)$ indicate the set of vertices at distance $l$ or smaller from vertex $i$. The probability that the history of vertex $i$ escapes $B(i,l)$ in time $s$ is bounded by
		\begin{equation}
			\prob\left[\bigcup_{u \in [0, s]}\mathcal{H}_i(t^* - u) \nsubseteq B(i, l)\right] \leq \exp\left(s \Delta^2 - l \ln \Delta \right).
		\end{equation}
	\end{lemma}
	\begin{proof}
		Let $\mathcal{W} = \{\vect{w} = (w_1, w_2, \dots, w_l) : w_1 = i, ||w_{k-1} - w_k|| = 1\}$ be the set of length $l$ sequences of adjacent vertices starting at vertex $i$. If $\mathcal{H}_i$ contains any vertex outside $B(i, l)$ at a time $u \in [t^* - s, t^*]$ then there must be some sequence $w \in \mathcal{W}$ such that each $w_i$ was updated at some time $t^* > t_i > t^* - s$ and $t_{k-1} > t_k$. Call this event $M_w$. For any particular sequence $w$,
		\begin{equation}
			\prob[M_w] = \prob[\mathrm{Po}(s) \geq l]
		\end{equation}
		where $\mathrm{Po}(s)$ is Poisson with rate $s$. By a union bound over $\mathcal{W}$,
		\begin{equation}
			\prob\left[\bigcup_{u \in [0, s]}\mathcal{H}_i(t^* - u) \nsubseteq B(i, l)\right] \leq \Delta^{l-1} \prob[\mathrm{Po}(s) \geq l].
		\end{equation}
		The moment generating function of a poisson random variable with rate $s$ is
		\begin{equation}
			M(t) = \exp\left(s\left(\euler^t - 1\right)\right).
		\end{equation}
		Using a Chernoff bound we have for every $t > 0$,
		\begin{equation}
			\prob[\mathrm{Po}(s) \geq l] \leq \exp\left(s\left(\euler^t - 1\right) - t l\right).
		\end{equation}
		Overall we have
		\begin{align}
			\prob\left[\bigcup_{u \in [0, s]}\mathcal{H}_i(t^* - u) \nsubseteq B(i, l)\right] &\leq \Delta^{l-1} \exp\left(s\left(\euler^t - 1\right) - t l\right)\\
			&\leq \exp\left(s\left(\euler^t - 1\right) + l (\ln \Delta - t) \right).
		\end{align}
		Choosing $t = 2 \ln \Delta$,
		\begin{align}
			\prob\left[\bigcup_{u \in [0, s]}\mathcal{H}_i(t^* - u) \nsubseteq B(i, l)\right] &\leq \exp\left(s\left(\Delta^2 - 1\right) - l \ln \Delta \right)\\
			&\leq \exp\left(s \Delta^2 - l \ln \Delta \right).
		\end{align}
	\end{proof}

% \begin{corollary}
% \label{cor:prob update function time 0 in B(i, l)}
% 	\begin{equation}
% 		\prob[\mathcal{H}_i(0) \nsubseteq B(i,l)] \leq \Delta^{-l} n^{\Delta^2} \exp(z \Delta^2)
% 	\end{equation}
% 	For $l \geq \log(n)(\Delta^2 + 1)/\log(\Delta)$
% 	\begin{equation}
% 			\prob[\mathcal{H}_i(0) \nsubseteq B(i,l)] \leq \frac{\exp(z \Delta^2)}{n}
% 	\end{equation}
% \end{corollary}

% \begin{lemma}
% 	Blah
% \end{lemma}
% \begin{proof}
% 	Write $\bar{\mathcal{H}}_i$ for the history at $i$ but it only has branching updates at rate 1. Write $d_j$ for the time of death of history $j$. Let $k = |i-j|$.
% 	\begin{align}
% 		\prob(\bar{\mathcal{H}}_i \cap \mathcal{H}_j) &= \prob(\bar{\mathcal{H}}_i \cap \mathcal{H}_j | d_j < c(k)) \prob(d_j < c(k)) \notag \\
% 		&\phantom{=}+ \prob(\bar{\mathcal{H}}_i \cap \mathcal{H}_j | d_j \geq c(k)) \prob(d_j \geq c(k))\\
% 		&\leq \prob(\bar{\mathcal{H}}_i \cap \mathcal{H}_j | d_j < c(k)) + \prob(d_j \geq c(k))\\
% 		&\leq \prob(\bar{\mathcal{H}}_i \cap \mathcal{H}_j | d_j < c(k)) + \euler^{-(1 - \beta \Delta) c(k)}\\
% 		&\leq 2\prob(\bar{\mathcal{H}}_i(t^* - c(k)) \nsubseteq B(i, k/2)) + \euler^{-(1 - \beta \Delta) c(k)}\\
% 		&\leq \exp\left(c(k) \Delta^2 - k/2 \ln \Delta \right) + \euler^{-(1 - \beta \Delta) c(k)}\\
% 	\end{align}
% 	Choosing $c(k) = \ln(k)^2$ works so long as $\beta \Delta < 1$.
% \end{proof}


% \begin{lemma}
% 	Let $i$ and $j$ be the indices of two vertices at distance $k$. Then
% 	\begin{equation}
% 		\prob[X_j = 1 | X_i = 1] \leq
% 	\end{equation}
% \end{lemma}
% \begin{proof}
% 	Let $B(i, l)$ denote the set of vertices within distance $l$ of vertex $i$. Let $A$ be the event
% 	\begin{equation}
% 		\{\mathcal{H}_i \subset B(i,k/2), \mathcal{H}_j \subset B(j, k/2)\}.
% 	\end{equation}
% 	By the law of total probability,
% 	\begin{align}
% 		\prob[X_j = 1 | X_i = 1] &= \prob[X_j = 1 | X_i = 1, A] \prob[A|X_i = 1] \notag \\ 
% 		&\phantom{=}  +\prob\left[X_j = 1 | X_i = 1, A^\complement\right] \prob\left[A^\complement|X_i = 1\right].
% 	\end{align}
% 	If we are conditioning on $A$, then the events $\{X_j = 1\}$ and $\{X_i = 1\}$ depend only on the update sequences within $B(j, k/2)$ and $B(i, k/2)$ respectively. This means that
% 	\begin{equation}
% 		\prob[X_j = 1 | X_i = 1, A] = \prob[X_j = 1 | A].
% 	\end{equation}
% 	So
% 	\begin{align}
% 		\prob[X_j = 1 | X_i = 1] &= \prob[X_j = 1 | A] \prob[A|X_i = 1] \notag \\ 
% 		&\phantom{=}  +\prob\left[X_j = 1 | X_i = 1, A^\complement\right] \prob\left[A^\complement|X_i = 1\right]\\
% 		&\leq \prob[X_j = 1 | A] + \prob\left[A^\complement | X_i = 1\right]
% 	\end{align}
% \end{proof}
% \begin{lemma}
% 	\begin{equation}
% 		\prob[X_j = 1 | A] \leq 
% 	\end{equation}
% \end{lemma}
% \begin{proof}
% 	\begin{align}
% 		\prob[X_j = 1 | A] &= \prob[X_j = 1 | \mathcal{H}_i \subset B(i, k/2), \mathcal{H}_j \subset B(j, k/2)]%\\
% 		% &\leq \prob[X_j = 1]/\prob[\mathcal{H}_i \subset B(i, k/2), \mathcal{H}_j \subset B(j, k/2)]
% 	\end{align}
% 	The events $\{\mathcal{H}_i \subset B(i, k/2)\}$ and $\{\mathcal{H}_j \subset B(j, k/2)\}$ depend only the update sequence strictly within $B(i, k/2)$ and $B(j, k/2)$ respectively. When we condition on $\{\mathcal{H}_j \subset B(j, k/2)\}$ the event $\{X_j = 1\}$ depends only on the update sequence strictly within $B(j, k/2)$. So
% 	\begin{align}
% 		\prob[X_j = 1 | A] &= \prob[X_j = 1 | \mathcal{H}_j \subset B(j, k/2)]\\
% 			&\leq m_{t^*} \prob[\mathcal{H}_j \subset B(j, k/2)]^{-1}\\
% 			&= \frac{m_{t^*}}{1 - \prob[\mathcal{H}_j \not\subset B(j, k/2)]}.
% 	\end{align}
% 	From Lemma \ref{lem:history escaping ball bound}, 
% 	\begin{align}
% 		\prob[X_j = 1 | A] &\leq \frac{m_{t^*}}{1 - \exp(\alpha -\lambda k/2)}\\
% 		&\leq \frac{m_{t^*}}{1 - \exp(\alpha - \lambda/2)}
% 	\end{align}
% 	since $k \geq 1$.

% 	% So these events are independent. Continuing
% 	% \begin{align}
% 	% 	\prob[X_j = 1 | A] &\leq m_{t^*}^{-1} \prob[\mathcal{H}_i \subset B(i, k/2)]^{-2}\\
% 	% 	&\sim \frac{1}{n} (1 - n^{\Delta^2} \Delta^{-k/2})^{-2}
% 	% \end{align}
% 	% blergh
% \end{proof}
% \begin{lemma}
% 	\begin{equation}
% 		\prob[A^\complement | X_i = 1] \leq 
% 	\end{equation}
% \end{lemma}
% \begin{proof}
% 	For ease of notation, let $A_i$ be the event
% 	\begin{equation}
% 		A_i = \{\mathcal{H}_i \subset B(j, k/2)\}.
% 	\end{equation}
% 	Obviously, 
% 	\begin{equation}
% 		A = A_i \cap A_j.
% 	\end{equation}
% 	Now
% 	\begin{align}
% 		\prob\left[A^\complement | X_i = 1\right] &\leq \prob\left[A_i^\complement | X_i = 1\right] + \prob\left[A_j^\complement | X_i = 1\right]\\
% 		&= \prob\left[A_i^\complement | X_i = 1\right] + \prob\left[A_j^\complement | X_i = 1, A_i^\complement \right] \prob\left[A_i^\complement | X_i = 1\right] \notag \\
% 		&\phantom{=} + \prob\left[A_j^\complement | X_i = 1, A_i \right] \prob\left[A_i | X_i = 1\right]\\
% 		&\leq 2\prob\left[A_i^\complement | X_i = 1\right]+ \prob\left[A_j^\complement | X_i = 1, A_i \right].\\
% 	\end{align}
% 	If we condition on $A_i$ then the event $X_i = 1$ depends only on the update sequence strictly inside $B(i, k/2)$. Also note that the event $A_j$ depends only on the update sequence stricly inside $B(j, k/2)$. These update sequences are independent as they do not share any vertices and so
% 	\begin{equation}
% 		\prob\left[A_j^\complement | X_i = 1, A_i \right] = \prob\left[A_j^\complement \right]
% 	\end{equation}
% 	and we have
% 	\begin{equation}
% 		\prob\left[A^\complement | X_i = 1\right] \leq 2\prob\left[A_i^\complement | X_i = 1\right]+ \prob\left[A_j^\complement \right].
% 	\end{equation}
% \end{proof}
% \begin{lemma}
% 	\begin{equation}
% 		\prob\left[\mathcal{H}_i \not\subset B(i, l) | X_i = 1\right] \leq
% 	\end{equation}
% \end{lemma}
% \begin{proof}
% 	\begin{align}
% 		\prob\left[\mathcal{H}_i \not\subset B(i, l) | X_i = 1\right] &= m_{t^*}^{-1} \prob\left[\mathcal{H}_i \not\subset B(i, l), X_i = 1\right]\\
% 		&\leq m_{t^*}^{-1} \prob\left[\chi(\mathcal{H}_i) \geq l, X_i = 1\right]\\
% 		&\leq m_{t^*}^{-1} \prob\left[\chi(\mathcal{H}_i) \geq l, Z(\mathcal{H}_i) \geq t^*\right]\\
% 		&= m_{t^*}^{-1} \expect\left[\indicator_{\chi(\mathcal{H}_i) \geq l} \indicator_{Z(\mathcal{H}_i) \geq t^*} \right]\\
% 		&\leq m_{t^*}^{-1} \expect\left[\indicator_{\chi(\mathcal{H}_i) \geq l} \indicator_{Z(\mathcal{H}_i) \geq \ln(n) + z} \right]\\
% 		&\leq m_{t^*}^{-1} \expect\left[\exp(\chi(\mathcal{H}_i) - l) \exp(Z(\mathcal{H}_i) - \ln(n) - z) \right]\\
% 		&= m_{t^*}^{-1} \exp(-l) \exp(-z - \ln(n)) \expect\left[\exp(\chi(\mathcal{H}_i) + Z(\mathcal{H}_i)) \right]\\
% 		&= \max(1, \exp(- \beta \Delta z)) \exp(-l) \expect\left[\exp(\chi(\mathcal{H}_i) + Z(\mathcal{H}_i)) \right]\\
% 		&\leq C_z \euler^{-l} BLAH
% 	\end{align}
% \end{proof}
\begin{lemma}
	There exists a constant $C > 0$ such that for sufficiently small $\beta$,
	\label{lem:nd X_j given X_i}
	\begin{equation}
		\prob[X_j = 1| X_i = 1] \leq m_{t^*} + C \euler^{-k}.
	\end{equation}
	where $k = |i - j|$ is the distance between vertices $i$ and $j$.
\end{lemma}
\begin{proof}
	There are two ways in which the update history of vertex $j$ can survive until time $0$. The update history can survive without intersecting with the update history of vertex $i$ or the update history of vertex $j$ can merge with the update history of vertex $i$ (whose survival we are conditioning on). %(Note that these events are not mutually exclusive as the update history of vertex $i+k$ could merge with the update history of vertex $i$ after it reaches time $0$.) 
	Breaking up the probability this way we have
	\begin{align}
		\prob[X_j = 1| X_i = 1] &= \prob[X_j = 1, \mathcal{H}_i \cap \mathcal{H}_j = \emptyset | X_i = 1] \notag\\
		&\phantom{=}+ \prob[X_j = 1, \mathcal{H}_i \cap \mathcal{H}_j \neq \emptyset | X_i = 1]\\
		&\leq \prob[X_j = 1, \mathcal{H}_i \cap \mathcal{H}_j = \emptyset | X_i = 1] + \prob[\mathcal{H}_i \cap \mathcal{H}_j \neq \emptyset | X_i = 1].
	\end{align}

	The result follows from Lemmas \ref{lem: prob X_j and no intersect given X_i = 1} and \ref{lem: prob intersect given X_i = 1}.
\end{proof}

\begin{lemma}
	\label{lem: prob X_j and no intersect given X_i = 1}
	Let $i$ and $j$ be the indices of two vertices on the cycle $(\mathbb{Z}/n\mathbb{Z})$. Then
	\begin{equation}
		\prob[X_j = 1, \hist_i \cap\hist_j = \emptyset|X_i = 1] \leq m_{t^*}
	\end{equation}
\end{lemma}
\begin{proof}
	To begin
	\begin{align}
		\prob[X_j = 1, \hist_i \cap\hist_j = \emptyset|X_i = 1] &= \prob[X_i = 1]^{-1}\prob[X_i = 1, X_j = 1, \hist_i \cap\hist_j = \emptyset]\\
		&= \frac{\prob[\hist_i(0) \cup \hist_j(0) \neq \emptyset, \hist_i \cap \hist_j = \emptyset]}{\prob[\hist_i(0) \neq \emptyset]}.
	\end{align}
	Define $S$ to be the first time that the histories intersect, or define $S = 0$ if the histories do not intersect before time $0$. We construct a new history, $\hist_j'$, in the following way. Along the interval $[S, t^*)$, $\hist_j'$ is constructed from the update sequence in the exact same way as $\hist_j$. However, along the interval $[0, S)$, we replace the update sequence with an another i.i.d. copy of the update sequence. Note that we have constructed $\hist_j'$ to be independent of $\hist_i$. Also note that in the event $\{\hist_i \cap \hist_j = \emptyset\}$, we have $\hist_j = \hist_j'$ and so
	\begin{align}
		\prob[\hist_i(0) \cup \hist_j(0) \neq \emptyset, \hist_i \cap \hist_j = \emptyset] &= \prob[\hist_i(0) \cup \hist_j'(0) \neq \emptyset, \hist_i \cap \hist_j = \emptyset]\\
		&\leq \prob[\hist_i(0) \cup \hist_j'(0) \neq \emptyset]\\
		&= \prob[\hist_i(0) \neq \emptyset] \prob[\hist_j'(0) \neq \emptyset]
	\end{align}
	since $\hist_i$ and $\hist_j'$ are independent. Since $\prob[\hist_j'(0) \neq \emptyset] = \prob[\hist_j(0) \neq \emptyset] = m_{t^*}$ we get the desired result.
\end{proof}

% \begin{lemma}
% 	\label{lem: prob X_j and no intersect given X_i = 1}
% 	There exists an $\alpha$ such that for $\beta$ sufficiently small
% 	\begin{equation}
% 		\prob[X_j = 1, \mathcal{H}_i \cap \mathcal{H}_j = \emptyset| X_i = 1] \leq C_{z, \epsilon} \exp(2 \alpha) n^{-\epsilon}.
% 	\end{equation}
% \end{lemma}
% \begin{proof}
% 	\begin{align}
% 		\prob[X_j = 1, \hist_i \cap \hist_j = \emptyset| X_i = 1] &= m_{t^*}^{-1} \prob[X_j = 1, X_i = 1, \hist_i \cap \hist_j = \emptyset]\\
% 		&\leq m_{t^*}^{-1} \prob\left[\frac{1}{2} \mathcal{L}(\hist_i \cup \hist_j) \geq t^*\right]\\
% 		&= m_{t^*}^{-1} \prob\left[\frac{1+\epsilon}{2} \mathcal{L}(\hist_i \cup \hist_j) \geq (1+\epsilon) t^*\right]
% 	\end{align}
% 	for any $0 < \epsilon < 1$. Continuing
% 	\begin{align}
% 		\prob[X_j = 1, \hist_i \cap \hist_j = \emptyset| X_i = 1] &\leq m_{t^*}^{-1} \expect\left[\indicator_{(1+\epsilon)\mathcal{L}(\hist_i \cup \hist_j)/2 \geq (1+\epsilon)t^*}\right]\\
% 		&= m_{t^*}^{-1} \expect\left[\exp((1+\epsilon)\mathcal{L}(\hist_i \cup \hist_j)/2 - (1+\epsilon)t^*)\right]\\
% 		&= m_{t^*}^{-1} \exp(-(1+\epsilon)t^*) \expect\left[\exp\left(\frac{1+\epsilon}{2}\mathcal{L}(\hist_i \cup \hist_j\right)\right].
% 	\end{align}
% 	Since $t^* \geq \ln(n) + z$, and $m_{t^*} \geq \min(\euler^{-z}, \euler^{-(1 - \beta\Delta)z})/n$,
% 	\begin{align}
% 		\prob[X_j = 1, \hist_i \cap \hist_j = \emptyset| X_i = 1] &\leq C_{z, \epsilon} n^{-\epsilon} \expect\left[\exp\left(\frac{1+\epsilon}{2}\mathcal{L}(\hist_i \cup \hist_j\right)\right]
% 	\end{align}
% 	where 
% 	\begin{equation}
% 		C_{z, \epsilon} = \exp(-\epsilon z) \max(1, \euler^{\beta\Delta z}).
% 	\end{equation}
% 	From Lemma \ref{lem:full submartingale thing}, there exists an $\alpha$ such that for $\beta$ sufficiently small
% 	\begin{equation}
% 		\prob[X_j = 1, \hist_i \cap \hist_j = \emptyset| X_i = 1] \leq C_{z, \epsilon} \exp(2 \alpha) n^{-\epsilon}.
% 	\end{equation}
% \end{proof}

This proof is closely based on that in \cite{Lubetzky2014-po}.
\begin{lemma}
	\label{lem: prob intersect given X_i = 1}
	Let $i$ and $j$ be the indices of two vertices separated by distance $k$. Then there exists a $C$ such that for sufficiently small $\beta$,
	\begin{equation}
		\prob[\mathcal{H}_i \cap \mathcal{H}_j \neq \emptyset | X_i = 1] \leq C\euler^{-k}.
	\end{equation}
\end{lemma}
\begin{proof}
	Firstly,
	\begin{align}
		\prob[\mathcal{H}_i \cap \mathcal{H}_j \neq \emptyset | X_i = 1] &= \frac{\prob[\left\{\mathcal{H}_i \cap \mathcal{H}_j \neq \emptyset\right\} \cap \{X_i = 1\}]}{\prob[X_i = 1]}\\
		&\leq m_{t^*}^{-1} \prob[\left\{\mathcal{H}_i \cap \mathcal{H}_j \neq \emptyset\right\} \cap \{X_i = 1\}]\\
		&\leq  m_{t^*}^{-1} \prob[\left\{\mathcal{H}_i \cap \mathcal{H}_j \neq \emptyset\right\} \cap \left(\{X_i = 1\} \cup \{X_j = 1\}\right)]\\
		&= m_{t^*}^{-1} \prob[\left\{\mathcal{H}_i \cap \mathcal{H}_j \neq \emptyset\right\} \cap \left\{\mathcal{H}_i(0) \cup \mathcal{H}_j(0) \neq \emptyset\right\}]
	\end{align}
	since
	\begin{equation}
		\left\{X_i = 1\right\} = \left\{\mathcal{H}_i(0) \neq \emptyset\right\}.
	\end{equation}
	% then
	% \begin{align}
	% 	\prob[\mathcal{H}_i \cap \mathcal{H}_j \neq \emptyset | X_i = 1] &\leq m_{t^*}^{-1}\prob\left[\left\{\hat{\mathcal{H}}_i \cap \hat{\mathcal{H}}_j \neq \emptyset \right\} \cap \left\{\mathcal{H}_i(0) \neq \emptyset\right\}\right],\\
	% 	&\leq m_{t^*}^{-1}\prob\left[\left\{\hat{\mathcal{H}}_i \cap \hat{\mathcal{H}}_j \neq \emptyset \right\} \cap \left\{\mathcal{H}_i(0) \cup \mathcal{H}_j(0) \neq \emptyset\right\}\right].
	% \end{align}
	% \begin{align}
	% 	\prob[\mathcal{H}_i \cap \mathcal{H}_j \neq \emptyset | X_i = 1] &\leq n C_z\prob\left[\left\{\hat{\mathcal{H}}_i \cap \hat{\mathcal{H}}_j \neq \emptyset \right\} \cap \left\{X_i = 1\right\}\right],\\
	% 	&\leq n C_z\prob\left[\left\{\hat{\mathcal{H}}_i \cap \hat{\mathcal{H}}_j \neq \emptyset \right\} \cap \left(\left\{X_i = 1\right\} \cup \left\{X_i = 1\right\}\right)\right].
	% \end{align}
	% Now since each update in our histories can never remove more than one vertex at a time, if $\hat{\mathcal{H}}_i(0) \cup \hat{\mathcal{H}}_j(0) = \emptyset$, then there must be some time $t > 0$ at which $\hat{\mathcal{H}}_i(t) \cup \hat{\mathcal{H}}_j(t) = \{w\}$ for a single vertex $w$. Proceeding backwards from $t^*$, define $S$ to be the first such time this happens, or put $S = 0$ if the histories always contain at least two vertices.
	Proceeding backwards from $t^*$, define $S$ to be the random time at which $\mathcal{H}_i(t) \cup \mathcal{H}_j(t)$ first reduced to less than two vertices, or define $S = 0$ if the combined histories contain at least two vertices all the way to time $0$. 
	Note that
	\begin{equation}
		\label{eq:prob union subset of}
		\{\hist_i(0) \cup \hist_j(0) \neq \emptyset\} \subseteq \{\mathscr{F}(v, 0, S) \neq \emptyset\}
	\end{equation}
	where $v$ is either the single vertex $v = \hist_i(S) \cup\hist_j(S)$ in the case that the histories coallesce to a single point at $S$, or any arbitrary single vertex otherwise.
	We also note that
	\begin{equation}
		\label{eq:prob intersect subset of}
		\{\hist_i \cap \hist_j \neq \emptyset\} \subseteq \{\chi\left((\hist_i \cup \hist_j)\cap V \times [S, t^*]\right) \geq k - 1\}
	\end{equation}
	since there must be at least $k - 1$ branching edges for the histories to meet. The event on the right hand side of \eqref{eq:prob intersect subset of} depends only on the update sequence in $[S, t^*]$. The event on the right hand side of \eqref{eq:prob union subset of} depends only on the update sequence in $[0, S)$. Therefore, given $S$, these events are independent and
	\begin{align}
		\prob[\left\{\mathcal{H}_i \cap \mathcal{H}_j \neq \emptyset\right\} \cap \left\{\mathcal{H}_i(0) \cup \mathcal{H}_j(0) \neq \emptyset\right\}| S = t_s] &\leq \prob[\mathscr{F}(v, 0, S) \neq \emptyset | S = t_s] \prob[\chi\left((\hist_i \cup \hist_j)\cap V \times [S, t^*]\right) \geq k - 1 | S = t_s]\\
		&= m_{t_s}  \prob[\chi\left((\hist_i \cup \hist_j)\cap V \times [S, t^*]\right) \geq k - 1 | S = t_s]
	\end{align}
	and so
	\begin{align}
		\prob[\left\{\mathcal{H}_i \cap \mathcal{H}_j \neq \emptyset\right\} \cap \left\{\mathcal{H}_i(0) \cup \mathcal{H}_j(0) \neq \emptyset\right\}] &\leq \expect[\indicator_{\chi\left((\hist_i \cup \hist_j)\cap V \times [S, t^*]\right) \geq k - 1 }m_S]\\
		&\leq \expect[\indicator_{\chi(\hist_i \cup \hist_j) \geq k - 1} m_S].
	\end{align}
	% If $\mathcal{H}_i(S) \cup \mathcal{H}_j(S)$ contains exactly one vertex then the probability of survival from time $S$ to time $0$ is simply the magnetisation $m_S$. Otherwise, either $S = 0$ or $|\mathcal{H}_i(S) \cup \mathcal{H}_j(S)| = 0$. In the latter case the histories cannot reach time $0$ and in the former, 
	% \begin{equation}
	% 	\prob\left[\mathcal{H}_i(0) \cup \mathcal{H}_j(0) \neq \emptyset | S = 0\right] = 1 = m_{0}.
	% \end{equation}
	% Overall,
	% \begin{equation}
	% 	\prob\left[\mathcal{H}_i(0) \cup \mathcal{H}_j(0) \neq \emptyset | S = t_s\right] \leq m_{t_s}.
	% \end{equation}
	From Corollary \ref{cor:exponential decay magnetization}, 
	\begin{equation}
		m_S \leq \euler^{t^* - S} m_{t^*}
	\end{equation}
	and since $|\hist_i(t) \cup \hist_j(t)| \geq 2$ for $t \in (S, t^*]$,
	\begin{equation}
		t^* - S \leq \mathcal{L}(\hist_i \cup \hist_j)/2.
	\end{equation}
	So
	\begin{align}
		\prob[\mathcal{H}_i \cap \mathcal{H}_j \neq \emptyset | X_i = 1] %&\leq m_{t^*}^{-1} \expect[\indicator_{\hat{\mathcal{H}}_i \cap \hat{\mathcal{H}}_j \neq \emptyset} m_S] \\
		% &\leq m_{t^*}^{-1} \expect[\indicator_{\chi(\mathcal{H}_{ij}) \geq k-1} m_S]\\
		&\leq m_{t^*}^{-1} m_{t^*} \expect[\indicator_{\chi(\hist_i \cup \hist_j) \geq k-1} \euler^{\mathcal{L}(\hist_i\cup \hist_j)/2}]\\
		% &= \expect[\indicator_{\chi(\mathcal{H}_{ij}) \geq k-1} \euler^{\mathcal{L}(\hist_i\cup \hist_j)/2}]\\
		&\leq \expect[\euler^{\chi(\hist_i\cup \hist_j) - (k-1)} \euler^{\mathcal{L}(\hist_i\cup \hist_j)/2}]\\
		% &\leq \euler^{-(k-1)} \expect[\euler^{\chi(\mathcal{H}_{ij}) + (1/2)L(\mathcal{H}_{ij}) + 1} ]\\
		&= \euler^{-k + 1} \expect[\euler^{\chi(\hist_i\cup \hist_j) + L(\hist_i\cup \hist_j)/2} ]
	\end{align}

	From Lemma \ref{lem:full submartingale thing}, there exists an $\alpha > \ln(2)$ such that for small enough $\beta$,
	\begin{align}
		\expect[\euler^{\chi(\hist_i\cup \hist_j) + L(\hist_i\cup \hist_j)/2} ] &\leq \euler^{2\alpha}
	\end{align}
	and we get the desired result by choosing $C = \exp(1 + 2\alpha)$.
\end{proof}